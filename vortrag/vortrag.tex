\documentclass[12pt,xcolor=dvipsnames,professionalfonts]{beamer}

% Pakete
\usepackage[utf8]{inputenc}
\usepackage[ngerman]{babel}

% AMS Pakete
\usepackage{amsmath}
\usepackage{amsfonts}
\usepackage{amssymb}

\usepackage{multirow}
\usepackage[percent]{overpic}

% Einheiten
\usepackage{siunitx}
\sisetup{
	output-decimal-marker={,},
	separate-uncertainty
}

% Grafiken
\usepackage{graphicx}
\usepackage{tabularx}
\setbeamerfont{caption}{size=\footnotesize}
\setbeamertemplate{caption}{\raggedright\insertcaption\par}

\newcommand{\todo}[1]{{\textcolor{Green}{(#1)}}}

% Theme
\usetheme{Boadilla}
\usecolortheme{rose}
\useoutertheme{infolines}
\useinnertheme{rectangles}
\setbeamertemplate{itemize subitem}[triangle]

\usefonttheme[onlymath]{serif}

% [num] Zitationen
\setbeamertemplate{bibliography item}[text]

% Navigationsleiste ausschalten
\beamertemplatenavigationsymbolsempty

\DeclareMathOperator{\divergence}{div}

\author[Christopher Deutsch]
{Christopher Deutsch}

\title
{Störkörpermessung an Hohlraumresonatoren}

\subtitle
{}
%\logo{}

\institute[]
{Rheinische Friedrich-Wilhelms-Universität Bonn \\
Seminar zur Bachelorarbeit SS15}

\date{24. September 2015}

%\setbeamercovered{transparent}
%\setbeamertemplate{navigation symbols}{}

\newcommand{\beginbackup}{
	\newcounter{framenumbervorappendix}
	\setcounter{framenumbervorappendix}{\value{framenumber}}
}
\newcommand{\backupend}{
	\addtocounter{framenumbervorappendix}{-\value{framenumber}}
	\addtocounter{framenumber}{\value{framenumbervorappendix}} 
}

\begin{document}
\maketitle

\begin{frame}{Inhalt}
	\tableofcontents
\end{frame}

\section{Motivation}
%\frame{\tableofcontents[currentsection]} Inhaltsverzeichnis für die aktuelle Section
% \setlength\itemsep{1em} in itemization zur abstandeinstellung
\begin{frame}{Motivation (Zuviel Text)}
	\begin{itemize}
		\item Begrenzung des internen Strahlstroms an ELSA durch fehlende Hochfrequenzleistung
		\begin{itemize}
			\setlength\itemsep{0.25em}
			\item max.\ interner Strahlstrom  $\sim\SI{20}{mA}$ bei \SI{3.2}{GeV}
		\end{itemize}
		\vfill
		
		\item Erweiterung des Stretcherrings durch zweite HF-Station
		\begin{itemize}
			\setlength\itemsep{0.25em}
			\item Klystron und zwei 7-zellige PETRA-Resonatoren
			\item interne Strahlströme bis zu \SI{200}{mA}
		\end{itemize}
		\vfill
		
		\item Bestimmung der elektrischen Feldverteilung durch resonante Störkörpermessung
		\begin{itemize}
			\setlength\itemsep{0.25em}
			\item Beschleunigungsspannung
			\item Beschleunigungseffizienz (Shuntimpedanz)
			\item Moden höherer Ordnung
		\end{itemize}
	\end{itemize}
\end{frame}


\section{Hohlraumresonatoren}

\subsection{Schwingungsmoden}
\begin{frame}{Schwingungsmoden}
	\begin{columns}[T]
		\column{0.60\textwidth}
		\begin{itemize}
			\item Hohlraum mit leitenden Wänden
			\begin{itemize}
				\setlength\itemsep{0.25em}
				
				\item stehende em.\ Wellen
				
				\item Randbedingungen:
				\begin{align*}
				E_\parallel = 0 \qquad B_\perp = 0
				\end{align*}
				
				\item erlauben nur bestimmte Feldkonfigurationen (Moden)
			\end{itemize}
		\end{itemize}
		\column{0.4\textwidth}
		\centering
		\includegraphics[scale=0.6]{./figures/pillbox.pdf}
	\end{columns}
	\vfill
	\begin{itemize}
		\item zylindersymmetrische Resonatormoden
		\begin{itemize}
			\setlength\itemsep{0.25em}
			\item TM und TE-Moden
			\item Bezeichnung durch drei Indizes $m, n, p$
		\end{itemize}
		
	\end{itemize}
\end{frame}

\begin{frame}[t]
	\begin{columns}[T]
		\column{0.5\textwidth}
		\begin{figure}[h]
			\centering
			\hspace*{0.70cm}\includegraphics[scale=0.4]{./figures/tm010.pdf}
			\vspace*{-0.2cm}
			\caption{$\mathrm{TM}_{010}$}
		\end{figure}
		
		\column{0.5\textwidth}
		\begin{figure}[h]
			\centering
			\hspace*{0.70cm}\includegraphics[scale=0.4]{./figures/tm210.pdf}
			\vspace*{-0.2cm}
			\caption{$\mathrm{TM}_{210}$}
		\end{figure}
	\end{columns}
	\vfill
	\begin{itemize}
		\item $\mathrm{TM}_{mnp}$ / $\mathrm{TE}_{mnp}$:
		\begin{itemize}
			\setlength\itemsep{0.25em}
			\item $m$: azimuthale Perioden
			\item $n$: radiale Knoten
			\item $p$: halbe longitudinale Perioden
		\end{itemize}
	\end{itemize}
\end{frame}


\subsection{elektrische Eigenschaften}
\begin{frame}{Impedanzmodell für Hohlraumresonatoren}
	\begin{columns}[c]
		\column{0.7\textwidth}
		\begin{itemize}
			\item Parallelschwingkreis als Modell für Hohlraumresonatoren
			\begin{itemize}
				\setlength\itemsep{0.25em}
				\item gültig in der Nähe einer Resonanz
				
				\item beschrieben durch Eigenfrequenz $\omega_0$, Kreisgüte $Q_0$, Shuntimpedanz $R_\mathrm{S}$
			\end{itemize}
		\end{itemize}
		
		\column{0.3\textwidth}
		\centering
		\includegraphics[scale=1.15]{./figures/RLC_circuit.pdf}
	\end{columns}
	\vfill
	\begin{columns}[T]
		\column{0.55\textwidth}
		\begin{itemize}
			\item Impedanzmodell:
			\begin{align*}
			Z(\omega) = \frac{R_\mathrm{S}}{1 + \mathrm{i} Q_0 \left( \frac{\omega}{\omega_0} - \frac{\omega_0}{\omega} \right)}
			\end{align*}
		\end{itemize}
		
		\column{0.45\textwidth}
		\begin{itemize}
			\item mittlere Verlustleistung:
			\begin{align*}
				P_\mathrm{V} = \frac{\left|U\right|^2}{2 R_\mathrm{S}}
			\end{align*}
		\end{itemize}
	\end{columns}
	

\end{frame}

\begin{frame}{Kopplung mit externer Leistungsquelle}
	\begin{columns}[c]
		\column{0.65\textwidth}
		\begin{itemize}
			\item (induktive) Schleifenkopplung
			\item Kopplung mit der äußeren Beschaltung
			\item Transformation der Impedanz
			\item $\kappa$
		\end{itemize}
		
		\column{0.35\textwidth}
		\centering
		\includegraphics[scale=0.8]{./figures/pillbox_loop.pdf}
	\end{columns}
	
\end{frame}

\begin{frame}
	\begin{itemize}
		\item komplexer Reflexionsfaktor:
		\begin{align*}
			\rho(\omega) = \frac{(\kappa - 1) + \mathrm{i} Q_0 \left(\frac{\omega}{\omega_0} - \frac{\omega_0}{\omega}\right) }{(\kappa + 1) + \mathrm{i} Q_0 \left( \frac{\omega}{\omega_0} - \frac{\omega_0}{\omega} \right)}
		\end{align*}
	\end{itemize}
	
	Bild von Resonanzkurven (evtl. Phase?)
\end{frame}
	



\section{Resonante Störkörpermessung}

\begin{frame}{Störkörpermessung}
	\begin{itemize}
		\item Störung des Feldes durch dielektrische oder magnetische (Stör-)Körper (BILD?)
		\item Abhängig von Feldamplitude am Ort des Störkörpers
	\end{itemize}
\end{frame}

\begin{frame}{Resonante Störkörpermessung}
	\begin{itemize}
		\item Verschiebung der Resonanzfrequenz (kleine Störung):
		\begin{align*}
			\frac{\Delta \omega}{\omega_0} = \frac{\int_{V} \, \mathrm{d}V \left( \vec{E}_0^* \cdot \vec{P} + \vec{B}_0^* \cdot \vec{M} \right)}{4 W_0}
		\end{align*}
	
	\item Polarisation und Magnetisierung abhängig vom Körper (Material und Form)
	
	\item erlaubt die Bestimmung der Felder des ungestörten Resonators
	
	\end{itemize}
\end{frame}

\section{Störkörpermessungen}
\begin{frame}{Aufbau}
	\includegraphics[width=1.\textwidth]{./figures/messaufbau.pdf}
\end{frame}
\begin{frame}{Störkörpermessungen}
	\vspace*{2cm}
	\centering
	\begin{overpic}[width=0.95\textwidth,tics=10]{./figures/messaufbau_refpos.pdf}
		\put (45,20) {
				\fcolorbox{Black}{White}{\input{./plots/resonanzkurve_ref.tex}}
			}
	\end{overpic}
\end{frame}

\begin{frame}{Störkörpermessungen}
	\addtocounter{framenumber}{-1} 
	\vspace*{2cm}
	\centering
	\begin{overpic}[width=0.95\textwidth,tics=10]{./figures/messaufbau_messpos.pdf}
		\put (45,20) {
			\fcolorbox{Black}{White}{\input{./plots/resonanzkurve.tex}}
		}
	\end{overpic}
\end{frame}

\section{Ergebnisse}
\begin{frame}{Ergebnisse}
	Ergebnisse
\end{frame}

\begin{frame}{Literatur}
	\begin{thebibliography}{9}
		\bibitem{foot}
		Christopher J. Foot,
		\emph{Atomic Physics},
		Oxford University Press 2005
		
	\end{thebibliography}
	
\end{frame}

\beginbackup
% Hier die Backupfolien
\begin{frame}{$p$}
	\centering
	\includegraphics[scale=1.0]{./figures/longitudinale_periode.pdf}
\end{frame}
\backupend

\end{document}