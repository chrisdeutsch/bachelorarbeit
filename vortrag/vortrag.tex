\documentclass[12pt,xcolor=dvipsnames,professionalfonts]{beamer}

% Pakete
\usepackage[utf8]{inputenc}
\usepackage[ngerman]{babel}

% AMS Pakete
\usepackage{amsmath}
\usepackage{amsfonts}
\usepackage{amssymb}

\usepackage{multirow}
\usepackage{layouts}

% Einheiten
\usepackage{siunitx}
\sisetup{
	output-decimal-marker={,},
	separate-uncertainty
}

% Grafiken
\usepackage{graphicx}
\usepackage{tabularx}
\setbeamerfont{caption}{size=\footnotesize}
\setbeamertemplate{caption}{\raggedright\insertcaption\par}

\newcommand{\todo}[1]{{\textcolor{Green}{(#1)}}}

% Theme
\usetheme{Boadilla}
\usecolortheme{rose}
\useoutertheme{infolines}
\useinnertheme{rectangles}
\setbeamertemplate{itemize subitem}[triangle]

\usefonttheme[onlymath]{serif}

% [num] Zitationen
\setbeamertemplate{bibliography item}[text]

% Navigationsleiste ausschalten
\beamertemplatenavigationsymbolsempty

\DeclareMathOperator{\divergence}{div}

\author[Christopher Deutsch]
{Christopher Deutsch}

\title
{Störkörpermessung an Hohlraumresonatoren}

\subtitle
{}
%\logo{}

\institute[]
{Rheinische Friedrich-Wilhelms-Universität Bonn \\
Seminar zur Bachelorarbeit SS15}

\date{24. September 2015}

%\setbeamercovered{transparent}
%\setbeamertemplate{navigation symbols}{}

\newcommand{\beginbackup}{
	\newcounter{framenumbervorappendix}
	\setcounter{framenumbervorappendix}{\value{framenumber}}
}
\newcommand{\backupend}{
	\addtocounter{framenumbervorappendix}{-\value{framenumber}}
	\addtocounter{framenumber}{\value{framenumbervorappendix}} 
}

\begin{document}
\maketitle

\begin{frame}{Inhalt}
	\tableofcontents
\end{frame}

\section{Einführung}
%\frame{\tableofcontents[currentsection]} Inhaltsverzeichnis für die aktuelle Section
% \setlength\itemsep{1em} in itemization zur abstandeinstellung
\subsection{Motivation}
\begin{frame}{Motivation}
	\begin{itemize}
		\item Begrenzung des internen Strahlstroms von ELSA durch fehlende Hochfrequenzleistung
		\item Erweiterung des Stretcherrings durch zwei siebenzellige Resonatoren (PETRA) (int.\ Strahlströme bis \SI{200}{mA})
		\item Störkörpermessung zur Bestimmung der Feldverteilung in den Resonatoren (Beschleunigungseffizienz \& HOM)
	\end{itemize}
\end{frame}


\section{Hohlraumresonatoren}

\subsection{Felder \& Moden}
\begin{frame}{Felder \& Moden}
	\begin{itemize}
		\item Hohlraum mit (ideal) leitenden Wänden
		
		\begin{itemize}
			\item stehende e.m. Wellen durch Randbedingungen
			\begin{align*}
				E_\parallel = 0 \qquad B_\perp = 0
			\end{align*}
			\item nur bestimmte Feldkonfigurationen (Moden) möglich (Maxwell. Gl.) (unterschiedliche Freq.)
		\end{itemize}
		
		\item In der Praxis zylindersymmetrische Resonatoren
		\begin{itemize}
			\item Unterscheidung von TM und TE-Moden
			\item Bezeichnung der Moden anhand von drei Indizes $n, m, p$ (Bilder?)
		\end{itemize}
		
	\end{itemize}
\end{frame}

\begin{frame}
	Rest zu Felder und Moden und Bilder der TM/TE Moden.
	Vllt Bild einer Pillbox Cavity auf der vorigen Seite.
\end{frame}


\subsection{elektrische Eigenschaften}
\begin{frame}{elektrische Eigenschaften von Hohlraumresonatoren}
	\begin{columns}[c]
		\column{0.6\textwidth}
		\begin{itemize}
			\item $RLC$-Parallelschwingkreis
			\begin{itemize}
				\item Eigenfrequenz $\omega_0$
				\item Kreisgüte $Q_0$
				\item Shuntimpedanz $R_\mathrm{S}$
			\end{itemize}
		\end{itemize}
		
		\column{0.4\textwidth}
		\centering
		\includegraphics[width=0.6\textwidth]{./figures/RLC_circuit.pdf}
	\end{columns}
	
	\begin{itemize}
		\item Impedanzmodell:
		\begin{align*}
			Z(\omega) = \frac{R_\mathrm{S}}{1 + \mathrm{i} Q_0 \left( \frac{\omega}{\omega_0} - \frac{\omega_0}{\omega} \right)}
		\end{align*}
	\end{itemize}
\end{frame}

\begin{frame}
	Schleifenkopplung
\end{frame}

\begin{frame}
	\begin{itemize}
		\item komplexer Reflexionsfaktor:
		\begin{align*}
			\rho(\omega) = \frac{(\kappa - 1) + \mathrm{i} Q_0 \left(\frac{\omega}{\omega_0} - \frac{\omega_0}{\omega}\right) }{(\kappa + 1) + \mathrm{i} Q_0 \left( \frac{\omega}{\omega_0} - \frac{\omega_0}{\omega} \right)}
		\end{align*}
	\end{itemize}
	
	Bild von Resonanzkurven (evtl. Phase?)
\end{frame}
	



\section{Resonante Störkörpermessung}

\begin{frame}{Störkörpermessung}
	\begin{itemize}
		\item Störung des Feldes durch dielektrische oder magnetische (Stör-)Körper (BILD?)
		\item Abhängig von Feldamplitude am Ort des Störkörpers
	\end{itemize}
\end{frame}

\begin{frame}{Resonante Störkörpermessung}
	\begin{itemize}
		\item Verschiebung der Resonanzfrequenz (kleine Störung):
		\begin{align*}
			\frac{\Delta \omega}{\omega_0} = \frac{\int_{V} \, \mathrm{d}V \left( \vec{E}_0^* \cdot \vec{P} + \vec{B}_0^* \cdot \vec{M} \right)}{4 W_0}
		\end{align*}
	
	\item Polarisation und Magnetisierung abhängig vom Körper (Material und Form)
	
	\item erlaubt die Bestimmung der Felder des ungestörten Resonators
	
	\end{itemize}
\end{frame}

\section{Störkörpermessungen}
\begin{frame}{Störkörpermessungen}
	Störkörpermessungen
	\put(5,1){\input{./plots/resonanzkurve.tex}}
\end{frame}

\section{Ergebnisse}
\begin{frame}{Ergebnisse}
	Ergebnisse
	\printinunitsof{cm}\prntlen{\textwidth}
\end{frame}

\begin{frame}{Literatur}
	\begin{thebibliography}{9}
		\bibitem{foot}
		Christopher J. Foot,
		\emph{Atomic Physics},
		Oxford University Press 2005
		
	\end{thebibliography}
	
\end{frame}

\beginbackup
% Hier die Backupfolien
\backupend

\end{document}