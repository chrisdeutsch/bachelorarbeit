\documentclass[12pt,xcolor=dvipsnames,professionalfonts]{beamer}

% Pakete
\usepackage[utf8]{inputenc}
\usepackage[ngerman]{babel}

% AMS Pakete
\usepackage{amsmath}
\usepackage{amsfonts}
\usepackage{amssymb}

\usepackage{multirow}

% Einheiten
\usepackage{siunitx}
\sisetup{
	output-decimal-marker={,},
	separate-uncertainty
}

% Grafiken
\usepackage{graphicx}
\usepackage{tabularx}
\setbeamerfont{caption}{size=\footnotesize}
\setbeamertemplate{caption}{\raggedright\insertcaption\par}

\newcommand{\todo}[1]{{\textcolor{Green}{(#1)}}}

% Theme
\usetheme{Boadilla}
\usecolortheme{rose}
\useoutertheme{infolines}
\useinnertheme{rectangles}
\setbeamertemplate{itemize subitem}[triangle]

\usefonttheme[onlymath]{serif}

% [num] Zitationen
\setbeamertemplate{bibliography item}[text]

% Navigationsleiste ausschalten
\beamertemplatenavigationsymbolsempty

\DeclareMathOperator{\divergence}{div}

\author[Christopher Deutsch]
{Christopher Deutsch}

\title
{Störkörpermessung an Hohlraumresonatoren}

\subtitle
{}
%\logo{}

\institute[]
{Rheinische Friedrich-Wilhelms-Universität Bonn \\
Seminar zur Bachelorarbeit SS15}

\date{24. September 2015}

%\setbeamercovered{transparent}
%\setbeamertemplate{navigation symbols}{}

\newcommand{\beginbackup}{
	\newcounter{framenumbervorappendix}
	\setcounter{framenumbervorappendix}{\value{framenumber}}
}
\newcommand{\backupend}{
	\addtocounter{framenumbervorappendix}{-\value{framenumber}}
	\addtocounter{framenumber}{\value{framenumbervorappendix}} 
}

\begin{document}
\maketitle

\begin{frame}{Inhalt}
	\tableofcontents
\end{frame}

\section{Einführung}
\frame{\tableofcontents[currentsection]}

\subsection{Motivation}


\section{Hohlraumresonatoren}
\frame{\tableofcontents[currentsection]}

\subsection{Felder \& Moden}
\begin{frame}{Hohlraumresonatoren}
	Hohlraumresonatoren
\end{frame}

\subsection{elektrische Eigenschaften}


\section{Resonante Störkörpermessung}
\frame{\tableofcontents[currentsection]}

\begin{frame}{Resonante Störkörpermessung}
	Resonante Störkörpermessung
\end{frame}

\section{Störkörpermessungen}
\begin{frame}{Störkörpermessungen}
	Störkörpermessungen
\end{frame}

\section{Ergebnisse}
\begin{frame}{Ergebnisse}
	Ergebnisse
\end{frame}

\begin{frame}{Literatur}
	\begin{thebibliography}{9}
		\bibitem{foot}
		Christopher J. Foot,
		\emph{Atomic Physics},
		Oxford University Press 2005
		
	\end{thebibliography}
	
\end{frame}

\beginbackup
% Hier die Backupfolien
\backupend

\end{document}