\documentclass[12pt,xcolor=dvipsnames,professionalfonts]{beamer}

% Pakete
\usepackage[utf8]{inputenc}
\usepackage[ngerman]{babel}

% AMS Pakete
\usepackage{amsmath}
\usepackage{amsfonts}
\usepackage{amssymb}

\usepackage{booktabs}
\usepackage{multirow}
\usepackage{array}
\usepackage[percent]{overpic}

% Einheiten
\usepackage{siunitx}
\sisetup{
	output-decimal-marker={,},
	separate-uncertainty
}

% Grafiken
\usepackage{graphicx}
\usepackage{tabularx}
\setbeamerfont{caption}{size=\footnotesize}
\setbeamertemplate{caption}{\raggedright\insertcaption\par}

\newcommand{\todo}[1]{{\textcolor{Green}{(#1)}}}

% Theme
\usetheme{Boadilla}
\usecolortheme{rose}
\useoutertheme{infolines}
\useinnertheme{rectangles}
\setbeamertemplate{itemize subitem}[triangle]

\usefonttheme[onlymath]{serif}

% [num] Zitationen
\setbeamertemplate{bibliography item}[text]

% Navigationsleiste ausschalten
\beamertemplatenavigationsymbolsempty

\DeclareMathOperator{\divergence}{div}

\author[Christopher Deutsch]
{Christopher Deutsch}

\title
{Störkörpermessung an Hohlraumresonatoren}

\subtitle
{}
%\logo{}

\institute[]
{Physikalisches Institut der Universität Bonn\\
Seminar zur Bachelorarbeit}

\date{24. September 2015}

%\setbeamercovered{transparent}
%\setbeamertemplate{navigation symbols}{}

\newcommand{\beginbackup}{
	\newcounter{framenumbervorappendix}
	\setcounter{framenumbervorappendix}{\value{framenumber}}
}
\newcommand{\backupend}{
	\addtocounter{framenumbervorappendix}{-\value{framenumber}}
	\addtocounter{framenumber}{\value{framenumbervorappendix}} 
}

\begin{document}
\maketitle

\begin{frame}{Inhalt}
	\tableofcontents
\end{frame}

\section{Motivation}
%\frame{\tableofcontents[currentsection]} Inhaltsverzeichnis für die aktuelle Section
% \setlength\itemsep{1em} in itemization zur abstandeinstellung
\begin{frame}{Motivation (Zuviel Text)}
	\begin{itemize}
		\item Begrenzung des internen Strahlstroms an ELSA durch fehlende Hochfrequenzleistung
		\begin{itemize}
			\setlength\itemsep{0.25em}
			\item max.\ interner Strahlstrom  $\sim\SI{20}{mA}$ bei \SI{3.2}{GeV}
		\end{itemize}
		\vfill
		
		\item Erweiterung des Stretcherrings durch zweite HF-Station
		\begin{itemize}
			\setlength\itemsep{0.25em}
			\item Klystron und zwei 7-zellige PETRA-Resonatoren
			\item interne Strahlströme bis zu \SI{200}{mA}
		\end{itemize}
		\vfill
		
		\item Bestimmung der elektrischen Feldverteilung durch resonante Störkörpermessung
		\begin{itemize}
			\setlength\itemsep{0.25em}
			\item Beschleunigungsspannung
			\item Beschleunigungseffizienz (Shuntimpedanz)
			\item Moden höherer Ordnung
		\end{itemize}
	\end{itemize}
\end{frame}


\section{Hohlraumresonatoren}

\subsection{Schwingungsmoden}
\begin{frame}{Schwingungsmoden}
	\begin{columns}[T]
		\column{0.60\textwidth}
		\begin{itemize}
			\item Hohlraum mit leitenden Wänden
			\begin{itemize}
				\setlength\itemsep{0.25em}
				
				\item stehende em.\ Wellen
				
				\item Randbedingungen:
				\begin{align*}
				E_\parallel = 0 \qquad B_\perp = 0
				\end{align*}
				
				\item erlauben nur bestimmte Feldkonfigurationen (Moden)
			\end{itemize}
		\end{itemize}
		\column{0.4\textwidth}
		\centering
		\includegraphics[scale=0.6]{./figures/pillbox.pdf}
	\end{columns}
	\vfill
	\begin{itemize}
		\item zylindersymmetrische Resonatormoden
		\begin{itemize}
			\setlength\itemsep{0.25em}
			\item TM und TE-Moden
			\item Bezeichnung durch drei Indizes $m, n, p$
		\end{itemize}
		
	\end{itemize}
\end{frame}

\begin{frame}[t]
	\begin{columns}[T]
		\column{0.5\textwidth}
		\begin{figure}[h]
			\centering
			\hspace*{0.70cm}\includegraphics[scale=0.4]{./figures/tm010.pdf}
			\vspace*{-0.2cm}
			\caption{$\mathrm{TM}_{010}$}
		\end{figure}
		
		\column{0.5\textwidth}
		\begin{figure}[h]
			\centering
			\hspace*{0.70cm}\includegraphics[scale=0.4]{./figures/tm210.pdf}
			\vspace*{-0.2cm}
			\caption{$\mathrm{TM}_{210}$}
		\end{figure}
	\end{columns}
	\vfill
	\begin{itemize}
		\item $\mathrm{TM}_{mnp}$ / $\mathrm{TE}_{mnp}$:
		\begin{itemize}
			\setlength\itemsep{0.25em}
			\item $m$: azimuthale Perioden
			\item $n$: radiale Knoten
			\item $p$: halbe longitudinale Perioden
		\end{itemize}
	\end{itemize}
\end{frame}


\subsection{elektrische Eigenschaften}
\begin{frame}{Impedanzmodell für Hohlraumresonatoren}
	\begin{columns}[c]
		\column{0.7\textwidth}
		\begin{itemize}
			\item Parallelschwingkreis als Modell für Hohlraumresonatoren
			\begin{itemize}
				\setlength\itemsep{0.25em}
				\item gültig in der Nähe einer Resonanz
				
				\item beschrieben durch Eigenfrequenz $\omega_0$, Kreisgüte $Q_0$, Shuntimpedanz $R_\mathrm{S}$
			\end{itemize}
		\end{itemize}
		
		\column{0.3\textwidth}
		\centering
		\includegraphics[scale=1.15]{./figures/RLC_circuit.pdf}
	\end{columns}
	\vfill
	\begin{columns}[T]
		\column{0.55\textwidth}
		\begin{itemize}
			\item Impedanzmodell:
			\begin{align*}
			Z(\omega) = \frac{R_\mathrm{S}}{1 + \mathrm{i} Q_0 \left( \frac{\omega}{\omega_0} - \frac{\omega_0}{\omega} \right)}
			\end{align*}
		\end{itemize}
		
		\column{0.45\textwidth}
		\begin{itemize}
			\item mittlere Verlustleistung:
			\begin{align*}
				P_\mathrm{V} = \frac{\left|U\right|^2}{2 R_\mathrm{S}}
			\end{align*}
		\end{itemize}
	\end{columns}
	

\end{frame}

\begin{frame}{Kopplung mit externer Leistungsquelle}
	\begin{columns}[c]
		\column{0.65\textwidth}
		\begin{itemize}
			\setlength\itemsep{1.25em}
			\item Übertragung von Energie in Resonatormode
			\begin{itemize}
				\setlength\itemsep{0.25em}
				\item induktive Schleifenkopplung
			\end{itemize}
			
			\item Kopplung an die äußere Beschaltung
			\begin{itemize}
				\setlength\itemsep{0.25em}
				\item Transformation der Impedanz $Z \rightarrow Z^\prime$
				\begin{align*}
					Z^\prime(\omega) = \kappa \frac{Z_0}{R_\mathrm{S}} Z(\omega)
				\end{align*}
			\end{itemize}
			
			\item Koppelfaktor $\kappa$:
			\begin{itemize}
				\setlength\itemsep{0.25em}
				\item Maß für Kopplung zwischen Wellenleiter und Resonator\setlength\itemsep{0.25em}
			\end{itemize}
		\end{itemize}
		
		\column{0.35\textwidth}
		\centering
		\includegraphics[scale=0.8]{./figures/pillbox_loop.pdf}
	\end{columns}
	
\end{frame}

\begin{frame}
	\begin{itemize}
		\item komplexer Reflexionsfaktor:
		\begin{align*}
		\rho(\omega) = \frac{(\kappa - 1) + \mathrm{i} Q_0 \left(\frac{\omega}{\omega_0} - \frac{\omega_0}{\omega}\right) }{(\kappa + 1) + \mathrm{i} Q_0 \left( \frac{\omega}{\omega_0} - \frac{\omega_0}{\omega} \right)}
		\end{align*}
	\end{itemize}
	\vfill
	\begin{columns}[c]
		\column{0.7\textwidth}
			\begin{small}
				\centering
				\input{./plots/resonanzkurve_2.tex}
			\end{small}		
		\column{0.3\textwidth}
		\vspace*{-1.3cm}
		\begin{flalign*}
			Q_0 &= 300 &\\
			\kappa &= 1{,}1 &
		\end{flalign*}
	\end{columns}
\end{frame}
	

\section{Resonante Störkörpermessung}

\begin{frame}{Resonante Störkörpermessung}
	\begin{itemize}
		\item Bestimmung von el.\ und mag.\ Feldamplituden in Resonatoren
	\end{itemize}
	\vspace*{0.12cm}
	\begin{columns}[c,onlytextwidth]
		\column{0.6\textwidth}
		\begin{itemize}
			\setlength\itemsep{1.25em}
			\item lokalisierte Störung der Felder durch diel.\ oder mag.\ Störkörper
			\begin{itemize}
				\item Polarisation / Magnetisierung durch äußeres Feld
			\end{itemize}
			
			\item Annahme: kleine Störkörper

			\item Störung abhängig von Feld am Ort des Störkörpers
		
	\end{itemize}
		\column{0.4\textwidth}
		\centering
		\includegraphics[scale=1.0]{./figures/stoerung.pdf}
		
	\end{columns}

\end{frame}

\begin{frame}
	\begin{itemize}
		\setlength\itemsep{1.25em}
		\item Verschiebung der Resonanzfrequenz:
		\begin{align*}
			\frac{\Delta \omega}{\omega_0} = -\frac{\int_{V} \, \mathrm{d}V \left[ \vec{E}_0^* \cdot \vec{P} + \vec{B}_0^* \cdot \vec{M} \right]}{4 W_0}
		\end{align*}
	
	\item Polarisation $\vec{P}$ und Magnetisierung $\vec{M}$ abhängig von:
		\begin{itemize}
			\setlength\itemsep{0.25em}
			\item Material und Form des Störkörpers
			\item Position des Störkörpers (\todo{ext.\ Feld})
		\end{itemize}
	\item Bestimmung der ortsabh.\ Feldamplituden des ungestörten Resonators möglich	
	\end{itemize}
\end{frame}

\begin{frame}
	\begin{itemize}
		\setlength\itemsep{1.25em}
		\item el.\ Feldamplitude für dielektrische Störkörper ($\vec{M} = 0$):
		\begin{align*}
			|\vec{E}_0(\vec{x}_\mathrm{s})| = \sqrt{- 4 \cdot \frac{W_0}{\alpha_\mathrm{s}} \cdot \frac{\Delta \omega}{\omega_0}} \quad \text{für} \quad \Delta \omega \leq 0
		\end{align*}
		
		\item Störkörperkonstante $\alpha_\mathrm{s}$ (abh.\ von Form, Permittivität)
	\end{itemize}
\end{frame}

\section{Störkörpermessungen}

\begin{frame}{PETRA-Resonator}
	\begin{figure}
		\centering
		\includegraphics[scale=0.6]{./figures/cavity.pdf}
	\end{figure}
	
	\begin{itemize}
		\setlength\itemsep{1.0em}
		\item Beschleunigung ultrarelativistischer Teilchen durch $\mathrm{TM}_{010}$-Mode bei $\nu_0 = \SI{499.67}{Hz}$
		
		\item Anregung der (gekoppelten) Zellen durch Koppelschleife
		
		\item Abstimmstempel zur Frequenzeinstellung
		
	\end{itemize}
\end{frame}

\begin{frame}{Störkörpermessstand}
	\begin{figure}
		\centering
		\includegraphics[width=1.\textwidth]{./figures/messaufbau.pdf}
	\end{figure}
	\begin{itemize}
		\setlength\itemsep{0.5em}
		\item Schrittmotor zur Bewegung des Störkörpers im Resonator
		\item vektorieller Netzwerkanalysator an Koppelschleife
		\begin{itemize}
			\item Anregung durch Hochfrequenzsignal
			\item Messung des (komplexen) Reflexionsfaktors
		\end{itemize}
		\item automatisierte Messung
	\end{itemize}	
\end{frame}

\begin{frame}{Problem}
	\begin{itemize}
		\setlength\itemsep{1.25em}
		\item Vermessung mit kleiner Schrittweite
		\begin{itemize}
			\setlength\itemsep{0.25em}
			\item Messdauer: $\sim \SI{2}{\hour}$
			\item Temperaturänderung nicht zu vermeiden
		\end{itemize}
		
		\item Resonanzfrequenz sensitiv auf Temperaturänderung: 
		\begin{align*}
			\frac{\Delta \nu}{\Delta T} \approx \SI{8}{\kilo\hertz\per\celsius}
		\end{align*}
		
		\item typische Frequenzverschiebung durch Störkörper:
		\begin{align*}
			\Delta \nu \sim \SI{-10}{\kilo\hertz}
		\end{align*}
		
		\item \textbf{Lösung:} messe Frequenzen relativ zu einer Referenzfrequenz
		
	\end{itemize}
\end{frame}


\begin{frame}{Messablauf}
	\vspace*{2cm}
	\centering
	\begin{overpic}[width=0.95\textwidth,tics=10]{./figures/messaufbau_refpos.pdf}
		\put (45,20) {
				\fcolorbox{Black}{White}{\input{./plots/resonanzkurve_ref.tex}}
			}
	\end{overpic}
\end{frame}

\begin{frame}{Messablauf}
	\addtocounter{framenumber}{-1} 
	\vspace*{2cm}
	\centering
	\begin{overpic}[width=0.95\textwidth,tics=10]{./figures/messaufbau_messpos.pdf}
		\put (45,20) {
			\fcolorbox{Black}{White}{\input{./plots/resonanzkurve.tex}}
		}
	\end{overpic}
\end{frame}

\section{Ergebnisse}
\begin{frame}{Ergebnisse}
	\begin{itemize}
		\setlength\itemsep{1.25em}
		\item Vermessung verschiedener Resonatormoden	
		\begin{itemize}
			\setlength\itemsep{0.25em}
			\item $\mathrm{TM}_{010}$-Moden
			\item Moden höherer Ordnung
		\end{itemize}
		
		\item Güte $Q_0$, Koppelfaktor $\kappa$, Resonanzfrequenz $\nu_0$ aus Reflexionsspektrum
		
		\item Berechnung des el. Feldes $E_0 / \sqrt{P_\mathrm{V}}$
		\begin{itemize}
			\setlength\itemsep{0.25em}
			\item unabhängig von eingekoppelter Leistung
			\item einfache Berechnung der Shuntimpedanz:
			\begin{align*}
			R_\mathrm{S} = \frac{|U|^2}{2 P_\mathrm{V}}
			\end{align*}
			
		\end{itemize}
	\end{itemize}
\end{frame}

\begin{frame}{$\text{TM}_{010}$-$\pi$-Mode}
	\centering
	\input{./plots/messung_pi_1.tex}
\end{frame}
\begin{frame}{$\text{TM}_{010}$-$\pi$-Mode}
	\addtocounter{framenumber}{-1}
	\centering
	\input{./plots/messung_pi_2.tex}
\end{frame}

\begin{frame}
	\centering
	{\def\arraystretch{2}\tabcolsep=10pt
	\begin{tabular}{p{3.5cm}|cc} \toprule
		$\mathrm{TM}_{010}$-$\pi$-Mode & \textbf{PETRA-III} & \textbf{PETRA-IV} \\ \midrule
		Kreisgüte $Q_0$ & \num{29560 +- 110} & \num{28200 +- 180} \\
		Shuntimpedanz $R_\mathrm{S}$ & \SI{25.7 +- 1.0}{\mega\ohm} & \SI{24.5 +- 0.9}{\mega\ohm} \\
		Beschleunigungs\-spannung~$U$\newline($P_\mathrm{V} = \SI{100}{\kilo\watt}$)& \SI{2.27 +- 0.05}{\mega\volt} & \SI{2.21 +- 0.04}{\mega\volt} \\
	\end{tabular}
	}
	\vspace*{0.5cm}
	\begin{itemize}
		\item \textbf{Vergleich:} fünfzellige PETRA-Resonatoren (\SI{100}{kW}) \cite{desy_petra5}
		\begin{align*}
			R_\mathrm{S} \approx \SI{15}{\mega\ohm} \qquad U \approx \SI{1.7}{\mega\volt}
		\end{align*}
	\end{itemize}
\end{frame}

\begin{frame}{Moden höherer Ordnung}
	\begin{itemize}
		\setlength\itemsep{1.25em}
		\item unendliche Anzahl von Moden höherer Ordnung
		\begin{itemize}
			\item Vorauswahl durch Simulationen \cite{schedler} und Güte
		\end{itemize}
		
		\item \textbf{Erwartung:} hohe Shuntimpedanz einer $\mathrm{TM}_{021}$-Mode
		\begin{itemize}
			\item größte Shuntimpedanz dieser Moden:
			\begin{align*}
			\nu_0 = \SI{1465.8}{\mega\hertz}\qquad R_\mathrm{S} = \SI{183.0 +- 6.4}{\kilo\ohm}
			\end{align*}
			\item zwei Größenordnungen unter der Beschleunigermode
		\end{itemize}
		
		\item höchste gemessene Shuntimpedanz:
		\begin{align*}
			\mathrm{TE}_{111}: \quad \nu_0 = \SI{702.7}{\mega\hertz} \qquad R_\mathrm{S} = \SI{326 +- 11}{\kilo\ohm}
		\end{align*}
	\end{itemize}
\end{frame}


\begin{frame}{Zusammenfassung}
	\begin{itemize}
		\setlength\itemsep{1.25em}
		\item Störkörpermessung zuverlässige Methode zur Feldvermessung
		
		\item Resonatoren erfüllen Erwartungen
		
		\item HOMS muss weiter überprüft werden (Strahlspektrum)
	\end{itemize}
\end{frame}

\begin{frame}{Literatur}
	\begin{thebibliography}{9}
		\bibitem{desy_petra5}
		DESY-MHFe,
		\emph{Data Sheet: 500 MHz, 5-Cell Cavity}
		
		\bibitem{schedler}
		M.\ Schedler et al.,
		\emph{A New RF station for the ELSA Stretcher Ring},
		IPAC'15 Conf.\ Proc.\ (2015)
		
			
	\end{thebibliography}
	
\end{frame}

\beginbackup
% Hier die Backupfolien
\begin{frame}{Longitudinale Schwingungsmoden $\mathrm{TM}_{01p}$}
	\centering
	\includegraphics[scale=1.0]{./figures/longitudinale_periode.pdf}
\end{frame}

\begin{frame}{Resonanzkurven}
	\begin{small}
		\centering
		\input{./plots/resonanzkurve_full.tex}
	\end{small}
\end{frame}

\begin{frame}{Störkörperkonstante $\alpha_\mathrm{s}$}
	\begin{itemize}
		\item Polarisation einer dielektrischen Kugel im homogenen el.\ Feld:
		\begin{align*}
			  \vec{P} = 3 \, \frac{\varepsilon_\mathrm{r} - 1}{\varepsilon_\mathrm{r} + 2} \, \varepsilon_0 \vec{E}_0
		\end{align*}
		
		\item resultierende Störkörperkonstante:
		\begin{align*}
			\alpha_\mathrm{s} = 3 \, \frac{\varepsilon_\mathrm{r} - 1}{\varepsilon_\mathrm{r} + 2} \, \varepsilon_0 V_\mathrm{s}
		\end{align*}
	\end{itemize}
\end{frame}

\begin{frame}{Moden einer Resonatorkette}
	\includegraphics[scale=0.9]{./figures/phasoren_top.pdf}
\end{frame}

\begin{frame}
	\includegraphics[scale=0.9]{./figures/phasoren_bottom.pdf}
\end{frame}
\backupend

\end{document}