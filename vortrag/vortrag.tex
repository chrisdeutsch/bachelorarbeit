\documentclass[12pt,xcolor=dvipsnames,professionalfonts]{beamer}

% Pakete
\usepackage[utf8]{inputenc}
\usepackage[ngerman]{babel}

% AMS Pakete
\usepackage{amsmath}
\usepackage{amsfonts}
\usepackage{amssymb}

\usepackage{booktabs}
\usepackage{multirow}
\usepackage{array}
\usepackage[percent]{overpic}

% Einheiten
\usepackage{siunitx}
\sisetup{
	output-decimal-marker={,},
	separate-uncertainty
}

% Grafiken
\usepackage{graphicx}
\usepackage{tabularx}
\setbeamerfont{caption}{size=\footnotesize}
\setbeamertemplate{caption}{\raggedright\insertcaption\par}

\newcommand{\todo}[1]{{\textcolor{Green}{(#1)}}}

% Theme
\usetheme{Boadilla}
\usecolortheme{rose}
\useoutertheme{infolines}
\useinnertheme{rectangles}
\setbeamertemplate{itemize subitem}[triangle]

\usefonttheme[onlymath]{serif}

% [num] Zitationen
\setbeamertemplate{bibliography item}[text]

% Navigationsleiste ausschalten
\beamertemplatenavigationsymbolsempty

\author[Christopher Deutsch]
{Christopher Deutsch}

\title
{Störkörpermessung an Hohlraumresonatoren}

\subtitle
{}
%\logo{}

\institute[]
{Physikalisches Institut der Universität Bonn\\
Seminar zur Bachelorarbeit}

\date{24. September 2015}

%\setbeamercovered{transparent}
%\setbeamertemplate{navigation symbols}{}

\newcommand{\beginbackup}{
	\newcounter{framenumbervorappendix}
	\setcounter{framenumbervorappendix}{\value{framenumber}}
}
\newcommand{\backupend}{
	\addtocounter{framenumbervorappendix}{-\value{framenumber}}
	\addtocounter{framenumber}{\value{framenumbervorappendix}} 
}

\begin{document}
\maketitle

\begin{frame}{Inhalt}
	\tableofcontents
\end{frame}

\section{Motivation}
%\frame{\tableofcontents[currentsection]} Inhaltsverzeichnis für die aktuelle Section
% \setlength\itemsep{1em} in itemization zur abstandeinstellung

\begin{frame}{Motivation}
	\centering
	\begin{overpic}[width=1.0\textwidth,tics=20]{./figures/elsa.pdf}
		\put (81,55) {
			\begin{minipage}[t]{0.3\textwidth}
				HF-1:\\
				$I_\mathrm{S} \approx \SI{20}{mA}$\\
				(bei \SI{3.2}{GeV})
			\end{minipage}
			}
		\put (33,5) {
			\begin{minipage}[t]{0.3\textwidth}
				HF-1 + HF-2:\\
				$I_\mathrm{S} \approx \SI{200}{mA}$\\
				(bei \SI{3.2}{GeV})
			\end{minipage}
			}
	\end{overpic}
\end{frame}


\section{Grundlagen}

\subsection{Hohlraumresonatoren}

\begin{frame}{Felder und Schwingungsmoden}
	\begin{columns}[T]
		\column{0.60\textwidth}
		\begin{itemize}
			\item Hohlraum mit leitenden Wänden
			\begin{itemize}
				\setlength\itemsep{0.25em}
				
				\item stehende em.\ Wellen
				
				\item Randbedingungen:
				\begin{align*}
				E_\parallel = 0 \qquad B_\perp = 0
				\end{align*}
				
				\item erlauben nur bestimmte Feldkonfigurationen (Moden)
			\end{itemize}
		\end{itemize}
		\column{0.4\textwidth}
		\centering
		\includegraphics[scale=0.6]{./figures/pillbox.pdf}
	\end{columns}
	\vfill
	\begin{itemize}
		\item zylindersymmetrische Resonatormoden
		\begin{itemize}
			\setlength\itemsep{0.25em}
			\item TM und TE-Moden
			\item Bezeichnung durch drei Indizes $m, n, p$
		\end{itemize}
		
	\end{itemize}
\end{frame}

\begin{frame}
	\begin{columns}[T]
		\column{0.5\textwidth}
		\begin{figure}[h]
			\centering
			\hspace*{0.70cm}\includegraphics[scale=0.4]{./figures/tm010.pdf}
			\vspace*{-0.2cm}
			\caption{$\mathrm{TM}_{010}$}
		\end{figure}
		
		\column{0.5\textwidth}
		\begin{figure}[h]
			\centering
			\hspace*{0.70cm}\includegraphics[scale=0.4]{./figures/tm210.pdf}
			\vspace*{-0.2cm}
			\caption{$\mathrm{TM}_{210}$}
		\end{figure}
	\end{columns}
	\vfill
	\begin{itemize}
		\item $\mathrm{TM}_{mnp}$ / $\mathrm{TE}_{mnp}$:
		\begin{itemize}
			\setlength\itemsep{0.25em}
			\item $m$: azimuthale Perioden
			\item $n$: radiale Knoten
			\item $p$: halbe longitudinale Perioden
		\end{itemize}
	\end{itemize}
\end{frame}

\begin{frame}{Impedanzmodell für Hohlraumresonatoren}
	\begin{columns}[c]
		\column{0.7\textwidth}
		\begin{itemize}
			\item Parallelschwingkreis als Modell für Hohlraumresonatoren
			\begin{itemize}
				\setlength\itemsep{0.25em}
				\item gültig in der Nähe einer Resonanz
				
				\item beschrieben durch Eigenfrequenz $\omega_0$, Kreisgüte $Q_0$, Shuntimpedanz $R_\mathrm{S}$
			\end{itemize}
		\end{itemize}
		
		\column{0.3\textwidth}
		\centering
		\includegraphics[scale=1.15]{./figures/RLC_circuit.pdf}
	\end{columns}
	\vfill
	\begin{itemize}
		\item mittlere Verlustleistung:
		\begin{align*}
		P_\mathrm{V} = \frac{\left|U\right|^2}{2 R_\mathrm{S}}
		\end{align*}
	\end{itemize}
\end{frame}

\begin{frame}{Kopplung mit externer Leistungsquelle}
	\begin{columns}[c]
		\column{0.65\textwidth}
		\begin{itemize}
			\item<2-> Ersatzschaltbild:
		\end{itemize}
		\vspace*{0.2cm}
		\centering
		\includegraphics<2>[scale=1.15]{./figures/RLC_coupling_step_1.pdf}
		\includegraphics<3>[scale=1.15]{./figures/RLC_coupling_step_2.pdf}
		\includegraphics<4>[scale=1.15]{./figures/RLC_coupling_step_3.pdf}
		\column{0.35\textwidth}
		\centering
		\includegraphics<1->[scale=0.8]{./figures/pillbox_loop.pdf}
	\end{columns}
	
\end{frame}

\begin{frame}{Resonanzkurve}
	\centering
	\begin{small}
		\input{./plots/resonanzkurve_step_1.tex}
	\end{small}
\end{frame}
\begin{frame}{Resonanzkurve}
	\addtocounter{framenumber}{-1}
	\centering
	\begin{small}
		\input{./plots/resonanzkurve_step_2.tex}
	\end{small}
\end{frame}
\begin{frame}{Resonanzkurve}
	\addtocounter{framenumber}{-1}
	\centering
	\begin{small}
		\input{./plots/resonanzkurve_step_3.tex}
	\end{small}
\end{frame}
\begin{frame}{Resonanzkurve}
	\addtocounter{framenumber}{-1}
	\centering
	\begin{small}
		\input{./plots/resonanzkurve_step_4.tex}
	\end{small}
\end{frame}

\subsection{Resonante Störkörpermessung}

\begin{frame}{Resonante Störkörpermessung}
	\begin{itemize}
		\item Bestimmung von el.\ und mag.\ Feldamplituden in Resonatoren
	\end{itemize}
	\vspace*{0.12cm}
	\begin{columns}[c,onlytextwidth]
		\column{0.6\textwidth}
		\begin{itemize}
			\setlength\itemsep{1.25em}
			\item lokalisierte Störung der Felder durch diel.\ oder mag.\ Störkörper
			\begin{itemize}
				\item Polarisation / Magnetisierung durch äußeres Feld
			\end{itemize}

			\item Störung abhängig von Feld am Ort des Störkörpers
			
			\item Annahme: kleine Störkörper
		
	\end{itemize}
		\column{0.4\textwidth}
		\centering
		\includegraphics[scale=1.0]{./figures/stoerung.pdf}
		
	\end{columns}

\end{frame}

\begin{frame}
	\begin{itemize}
		\setlength\itemsep{1.25em}
		\item Verschiebung der Resonanzfrequenz:
		\begin{align*}
			\frac{\Delta \omega}{\omega_0} = -\frac{\int_{V} \, \mathrm{d}V \left[ \textcolor{Green}{\vec{E}_0^*} \cdot \textcolor{Red}{\vec{P}} + \textcolor{Green}{\vec{B}_0^*} \cdot \textcolor{Red}{\vec{M}} \right]}{4 W_0}
		\end{align*}
	
	\item Polarisation $\textcolor{Red}{\vec{P}}$ und Magnetisierung $\textcolor{Red}{\vec{M}}$ abhängig von:
		\begin{itemize}
			\setlength\itemsep{0.25em}
			\item Material und Form des Störkörpers
			\item Position des Störkörpers
		\end{itemize}
	\item Bestimmung der ortsabh.\ Feldamplituden des ungestörten Resonators möglich	
	\end{itemize}
\end{frame}

\begin{frame}
	\begin{itemize}
		\setlength\itemsep{1.25em}
		\item el.\ Feldamplitude für dielektrische Störkörper ($\vec{M} = 0$):
		\begin{align*}
			|\vec{E}_0(\textcolor{Red}{\vec{x}_\mathrm{s}})| = \sqrt{- 4 \cdot \frac{W_0}{\textcolor{Red}{\alpha_\mathrm{s}}} \cdot \frac{\textcolor{Red}{\Delta \omega}}{\omega_0}} \quad \text{für} \quad \Delta \omega \leq 0
		\end{align*}
		
		\item Störkörperkonstante $\alpha_\mathrm{s}$ (abh.\ von Form, Permittivität)
	\end{itemize}
\end{frame}

\section{Störkörpermessungen an PETRA-Resonatoren}

\begin{frame}{PETRA-Resonator}
	\begin{figure}
		\centering
		\includegraphics[scale=0.6]{./figures/cavity.pdf}
	\end{figure}
	
	\begin{itemize}
		\setlength\itemsep{1.0em}
		\item Beschleunigung ultrarelativistischer Teilchen durch $\mathrm{TM}_{010}$-Mode bei $\nu_0 = \SI{499.67}{MHz}$
		
		\item Anregung der (gekoppelten) Zellen durch Koppelschleife
		
		\item Abstimmstempel zur Frequenzeinstellung
		
	\end{itemize}
\end{frame}

\begin{frame}{Störkörpermessstand}
	\begin{figure}
		\centering
		\includegraphics[width=1.\textwidth]{./figures/messaufbau.pdf}
	\end{figure}
	\begin{itemize}
		\setlength\itemsep{0.5em}
		\item Schrittmotor zur Bewegung des Störkörpers im Resonator
		\item vektorieller Netzwerkanalysator an Koppelschleife
		\begin{itemize}
			\item Anregung durch Hochfrequenzsignal
			\item Messung des (komplexen) Reflexionsfaktors
		\end{itemize}
		\item automatisierte Messung
	\end{itemize}	
\end{frame}

\begin{frame}{Temperaturabhängigkeit}
	\begin{itemize}
		\setlength\itemsep{1.25em}
		\item Vermessung mit kleiner Schrittweite
		\begin{itemize}
			\setlength\itemsep{0.25em}
			\item Messdauer: $\sim \SI{2}{\hour}$
			\item Temperaturänderung nicht zu vermeiden
		\end{itemize}
		
		\item Resonanzfrequenz sensitiv auf Temperaturänderung: 
		\begin{align*}
			\frac{\Delta \nu}{\Delta T} \approx \SI{8}{\kilo\hertz\per\celsius}
		\end{align*}
		
		\item typische Frequenzverschiebung durch Störkörper:
		\begin{align*}
			\Delta \nu \sim \SI{10}{\kilo\hertz}
		\end{align*}
		
		\item \textbf{Lösung:} messe Frequenzen relativ zu einer Referenzfrequenz
		
	\end{itemize}
\end{frame}


\begin{frame}{Messablauf}
	\vspace*{2cm}
	\centering
	\begin{overpic}[width=0.95\textwidth,tics=10]{./figures/messaufbau_refpos.pdf}
		\put (45,20) {
				\fcolorbox{Black}{White}{\input{./plots/resonanzkurve_ref.tex}}
			}
	\end{overpic}
\end{frame}

\begin{frame}{Messablauf}
	\addtocounter{framenumber}{-1} 
	\vspace*{2cm}
	\centering
	\begin{overpic}[width=0.95\textwidth,tics=10]{./figures/messaufbau_messpos.pdf}
		\put (45,20) {
			\fcolorbox{Black}{White}{\input{./plots/resonanzkurve.tex}}
		}
	\end{overpic}
\end{frame}

\section{Ergebnisse}
\begin{frame}{Ergebnisse}
	\begin{itemize}
		\setlength\itemsep{1.25em}
		\item Vermessung verschiedener Resonatormoden	
		\begin{itemize}
			\setlength\itemsep{0.25em}
			\item $\mathrm{TM}_{010}$-Moden
			\item Moden höherer Ordnung
		\end{itemize}
		
		\item Güte $Q_0$, Koppelfaktor $\kappa$, Resonanzfrequenz $\nu_0$ aus Reflexionsspektrum
		
		\item Berechnung des el. Feldes $E_0 / \sqrt{P_\mathrm{V}}$
		\begin{itemize}
			\setlength\itemsep{0.25em}
			\item unabhängig von eingekoppelter Leistung
			\item einfache Berechnung der Shuntimpedanz:
			\begin{align*}
			R_\mathrm{S} = \frac{|U|^2}{2 P_\mathrm{V}}
			\end{align*}
			
		\end{itemize}
	\end{itemize}
\end{frame}

\begin{frame}{$\text{TM}_{010}$-Beschleunigungsmode}
	\centering
	\input{./plots/frequenzverschiebung.tex}
\end{frame}
\begin{frame}{$\text{TM}_{010}$-Beschleunigungsmode}
	\centering
	\input{./plots/messung_pi_1.tex}
\end{frame}
\begin{frame}{$\text{TM}_{010}$-Beschleunigungsmode}
	\addtocounter{framenumber}{-1}
	\centering
	\input{./plots/messung_pi_2.tex}
\end{frame}

\begin{frame}
	\centering
	{\def\arraystretch{2}\tabcolsep=10pt
	\begin{tabular}{p{3.5cm}|cc} \toprule
		& \textbf{PETRA-III} & \textbf{PETRA-IV} \\ \midrule
		Kreisgüte $Q_0$ & \num{29560 +- 110} & \num{28200 +- 180} \\
		Shuntimpedanz $R_\mathrm{S}$ & \SI{25.7 +- 1.0}{\mega\ohm} & \SI{24.5 +- 0.9}{\mega\ohm} \\
		Beschleunigungs\-spannung~$U$\newline($P_\mathrm{V} = \SI{100}{\kilo\watt}$)& \SI{2.27 +- 0.05}{\mega\volt} & \SI{2.21 +- 0.04}{\mega\volt} \\
	\end{tabular}
	}
	\vspace*{0.5cm}
	\begin{itemize}
		\item \textbf{Vergleich:} fünfzellige PETRA-Resonatoren (\SI{100}{kW})
		\begin{align*}
			R_\mathrm{S} \approx \SI{15}{\mega\ohm} \qquad U \approx \SI{1.7}{\mega\volt}
		\end{align*}
	\end{itemize}
\end{frame}

\begin{frame}{Moden höherer Ordnung}
	\begin{itemize}
		\setlength\itemsep{1.25em}
		\item unbegrenzte Anzahl von Moden höherer Ordnung
		\begin{itemize}
			\item Vorauswahl durch Simulationen und Güte
		\end{itemize}
		
		\item \textbf{Erwartung:} hohe Shuntimpedanz einer $\mathrm{TM}_{021}$-Mode
		\begin{itemize}
			\item größte Shuntimpedanz dieser Moden:
			\begin{align*}
			\nu_0 = \SI{1465.8}{\mega\hertz}\qquad R_\mathrm{S} = \SI{183.0 +- 6.4}{\kilo\ohm}
			\end{align*}
			\item zwei Größenordnungen unter der Beschleunigermode
		\end{itemize}
		
		\item höchste gemessene Shuntimpedanz:
		\begin{align*}
			\mathrm{TE}_{111}: \quad \nu_0 = \SI{702.7}{\mega\hertz} \qquad R_\mathrm{S} = \SI{326 +- 11}{\kilo\ohm}
		\end{align*}
	\end{itemize}
\end{frame}

\begin{frame}{Impedanzspektrum}
	\centering
	% GNUPLOT: LaTeX picture with Postscript
\begingroup
  \makeatletter
  \providecommand\color[2][]{%
    \GenericError{(gnuplot) \space\space\space\@spaces}{%
      Package color not loaded in conjunction with
      terminal option `colourtext'%
    }{See the gnuplot documentation for explanation.%
    }{Either use 'blacktext' in gnuplot or load the package
      color.sty in LaTeX.}%
    \renewcommand\color[2][]{}%
  }%
  \providecommand\includegraphics[2][]{%
    \GenericError{(gnuplot) \space\space\space\@spaces}{%
      Package graphicx or graphics not loaded%
    }{See the gnuplot documentation for explanation.%
    }{The gnuplot epslatex terminal needs graphicx.sty or graphics.sty.}%
    \renewcommand\includegraphics[2][]{}%
  }%
  \providecommand\rotatebox[2]{#2}%
  \@ifundefined{ifGPcolor}{%
    \newif\ifGPcolor
    \GPcolortrue
  }{}%
  \@ifundefined{ifGPblacktext}{%
    \newif\ifGPblacktext
    \GPblacktexttrue
  }{}%
  % define a \g@addto@macro without @ in the name:
  \let\gplgaddtomacro\g@addto@macro
  % define empty templates for all commands taking text:
  \gdef\gplbacktext{}%
  \gdef\gplfronttext{}%
  \makeatother
  \ifGPblacktext
    % no textcolor at all
    \def\colorrgb#1{}%
    \def\colorgray#1{}%
  \else
    % gray or color?
    \ifGPcolor
      \def\colorrgb#1{\color[rgb]{#1}}%
      \def\colorgray#1{\color[gray]{#1}}%
      \expandafter\def\csname LTw\endcsname{\color{white}}%
      \expandafter\def\csname LTb\endcsname{\color{black}}%
      \expandafter\def\csname LTa\endcsname{\color{black}}%
      \expandafter\def\csname LT0\endcsname{\color[rgb]{1,0,0}}%
      \expandafter\def\csname LT1\endcsname{\color[rgb]{0,1,0}}%
      \expandafter\def\csname LT2\endcsname{\color[rgb]{0,0,1}}%
      \expandafter\def\csname LT3\endcsname{\color[rgb]{1,0,1}}%
      \expandafter\def\csname LT4\endcsname{\color[rgb]{0,1,1}}%
      \expandafter\def\csname LT5\endcsname{\color[rgb]{1,1,0}}%
      \expandafter\def\csname LT6\endcsname{\color[rgb]{0,0,0}}%
      \expandafter\def\csname LT7\endcsname{\color[rgb]{1,0.3,0}}%
      \expandafter\def\csname LT8\endcsname{\color[rgb]{0.5,0.5,0.5}}%
    \else
      % gray
      \def\colorrgb#1{\color{black}}%
      \def\colorgray#1{\color[gray]{#1}}%
      \expandafter\def\csname LTw\endcsname{\color{white}}%
      \expandafter\def\csname LTb\endcsname{\color{black}}%
      \expandafter\def\csname LTa\endcsname{\color{black}}%
      \expandafter\def\csname LT0\endcsname{\color{black}}%
      \expandafter\def\csname LT1\endcsname{\color{black}}%
      \expandafter\def\csname LT2\endcsname{\color{black}}%
      \expandafter\def\csname LT3\endcsname{\color{black}}%
      \expandafter\def\csname LT4\endcsname{\color{black}}%
      \expandafter\def\csname LT5\endcsname{\color{black}}%
      \expandafter\def\csname LT6\endcsname{\color{black}}%
      \expandafter\def\csname LT7\endcsname{\color{black}}%
      \expandafter\def\csname LT8\endcsname{\color{black}}%
    \fi
  \fi
    \setlength{\unitlength}{0.0500bp}%
    \ifx\gptboxheight\undefined%
      \newlength{\gptboxheight}%
      \newlength{\gptboxwidth}%
      \newsavebox{\gptboxtext}%
    \fi%
    \setlength{\fboxrule}{0.5pt}%
    \setlength{\fboxsep}{1pt}%
\begin{picture}(6802.00,4534.00)%
    \gplgaddtomacro\gplbacktext{%
      \csname LTb\endcsname%
      \put(1210,704){\makebox(0,0)[r]{\strut{}\SI{1}{\kilo\ohm}}}%
      \csname LTb\endcsname%
      \put(1210,1417){\makebox(0,0)[r]{\strut{}\SI{10}{\kilo\ohm}}}%
      \csname LTb\endcsname%
      \put(1210,2130){\makebox(0,0)[r]{\strut{}\SI{100}{\kilo\ohm}}}%
      \csname LTb\endcsname%
      \put(1210,2843){\makebox(0,0)[r]{\strut{}\SI{1}{\mega\ohm}}}%
      \csname LTb\endcsname%
      \put(1210,3556){\makebox(0,0)[r]{\strut{}\SI{10}{\mega\ohm}}}%
      \csname LTb\endcsname%
      \put(1210,4269){\makebox(0,0)[r]{\strut{}\SI{100}{\mega\ohm}}}%
      \csname LTb\endcsname%
      \put(1583,484){\makebox(0,0){\strut{}$500$}}%
      \csname LTb\endcsname%
      \put(2547,484){\makebox(0,0){\strut{}$700$}}%
      \csname LTb\endcsname%
      \put(3512,484){\makebox(0,0){\strut{}$900$}}%
      \csname LTb\endcsname%
      \put(4476,484){\makebox(0,0){\strut{}$1100$}}%
      \csname LTb\endcsname%
      \put(5441,484){\makebox(0,0){\strut{}$1300$}}%
      \csname LTb\endcsname%
      \put(6405,484){\makebox(0,0){\strut{}$1500$}}%
    }%
    \gplgaddtomacro\gplfronttext{%
      \csname LTb\endcsname%
      \put(176,2486){\rotatebox{-270}{\makebox(0,0){\strut{}$R_\mathrm{S}$}}}%
      \put(3873,154){\makebox(0,0){\strut{}$\nu$ / \si{MHz}}}%
      \csname LTb\endcsname%
      \put(5682,4096){\makebox(0,0)[r]{\strut{}PETRA-III}}%
      \csname LTb\endcsname%
      \put(5682,3876){\makebox(0,0)[r]{\strut{}PETRA-IV}}%
      \csname LTb\endcsname%
      \put(1607,3969){\makebox(0,0){\strut{}\scriptsize{$\mathrm{TM}_{010}$}}}%
      \put(2557,2628){\makebox(0,0){\strut{}\scriptsize{$\mathrm{TE}_{111}$}}}%
      \put(2692,2020){\makebox(0,0){\strut{}\scriptsize{$\mathrm{TM}_{011}$}}}%
      \put(4221,1101){\makebox(0,0){\strut{}\scriptsize{$\mathrm{TM}_{111}$}}}%
      \put(6164,2449){\makebox(0,0){\strut{}\scriptsize{$\mathrm{TM}_{021}$}}}%
    }%
    \gplbacktext
    \put(0,0){\includegraphics{./plots/impedanzspektrum}}%
    \gplfronttext
  \end{picture}%
\endgroup

\end{frame}

\begin{frame}{Zusammenfassung}
	\begin{itemize}
		\setlength\itemsep{1.25em}
		\item Aufbau eines zuverlässigen Messstandes
		\begin{itemize}
			\item ohne Temperaturstabilisierung
			\item vollständige Automatisierung der Messung
		\end{itemize}
		
		\item Charakterisierung der $\mathrm{TM}_{010}$-Beschleunigermode beider Resonatoren
		
		\item Vermessung von Moden höherer Ordnung
	\end{itemize}
\end{frame}

\begin{frame}{Vielen Dank für die Aufmerksamkeit}
\end{frame}

\beginbackup
% Hier die Backupfolien

\begin{frame}
	\centering
	\textbf{Zusätzliche Folien:}
\end{frame}

\begin{frame}{Resonatormoden $\mathrm{TM}_{01p}$}
	\centering
	\includegraphics[scale=1.0]{./figures/longitudinale_periode.pdf}
\end{frame}

\begin{frame}{Störkörperkonstante $\alpha_\mathrm{s}$}
	\begin{itemize}
		\item Polarisation einer dielektrischen Kugel im homogenen el.\ Feld:
		\begin{align*}
			  \vec{P} = 3 \, \frac{\varepsilon_\mathrm{r} - 1}{\varepsilon_\mathrm{r} + 2} \, \varepsilon_0 \vec{E}_0
		\end{align*}
		
		\item resultierende Störkörperkonstante:
		\begin{align*}
			\alpha_\mathrm{s} = 3 \, \frac{\varepsilon_\mathrm{r} - 1}{\varepsilon_\mathrm{r} + 2} \, \varepsilon_0 V_\mathrm{s}
		\end{align*}
		
		\item dielektrische Kugel mit $R = \SI{1}{\centi\metre}$, $\varepsilon_\mathrm{r} = \num{2.1+- 0.05}$ (PTFE):
		\begin{align*}
			\alpha_\mathrm{s} = \SI{2.99 +- 0.11e-17}{\ampere\second\metre\squared\per\volt}
		\end{align*}
	\end{itemize}
\end{frame}

\begin{frame}{Moden einer Resonatorkette}
	\includegraphics[scale=0.9]{./figures/phasoren_top.pdf}
\end{frame}

\begin{frame}
	\includegraphics[scale=0.9]{./figures/phasoren_bottom.pdf}
\end{frame}
\backupend

\end{document}