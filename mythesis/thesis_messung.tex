%------------------------------------------------------------------------------
\chapter{Störkörpermessung}
\label{sec:stoerkoerpermessung}
%------------------------------------------------------------------------------
\section{Vorbereitung}
\subsection{Vorbereitung}
Abstimmen auf Luft, Symmetrisierung des Feldes

Unterschied Vakuum--Luft \cite{pozar} (S. 308):
\begin{align}
\omega \approx \omega_0 \cdot \frac{3 - \epsilon_\mathrm{r}}{2}
\end{align}
Permittivität Luft (trocken und CO2-frei) \cite{CRC} (S.1093):
\begin{align}
\epsilon_\mathrm{r}^\mathrm{Luft} = \num{1.0005364}
\end{align}



%------------------------------------------------------------------------------
\subsection{Störkörperkonstante}
%------------------------------------------------------------------------------

Zur Berechnung der Störkörperkonstante~$\alpha_\mathrm{s}$ des kugelförmigen Störkörpers aus PTFE kann Gleichung \eqref{eq:stoerkoerperkonstante} verwendet werden.
Die relative Permittivität von PTFE beträgt im Mittel $\varepsilon_\mathrm{r} = \num{2.1}$ und zeigt keine Frequenzabhängigkeit bis Frequenzen der Größenordnung \SI{1}{GHz} \cite[S.\ 2201]{CRC}.
Außerdem variiert die relative Permittivität mit der Dichte und Kristallinität des Materials, weshalb für den verwendeten Werkstoff $\varepsilon_\mathrm{r} = \num{2.1 +- 0.05}$ angenommen werden muss.
Unter Verwendung der Dimensionen des Störkörpers (vgl.\ \ref{sec:aufbau_messstand}) folgt die Störkörperkonstante
\begin{align}
	\alpha_\mathrm{s} = \SI{2.99 +- 0.11e-17}{\ampere\second\metre\squared\per\volt} \eqcomma
	\label{eq:stoerkoerperkonstante_ptfe}
\end{align}
wobei die zentrische Bohrung bei der Berechnung beachtet wurde.

\todo{wirkt als statistischer Fehler}

%------------------------------------------------------------------------------
\subsection{Bestimmung von Resonanzfrequenz und Güte}
\label{sec:resfreq_guete}
%------------------------------------------------------------------------------
Die Bestimmung von Resonanzfrequenz und Güte der Resonatormoden kann anhand des gemessenen Reflexionsspektrums der jeweiligen Resonanz erfolgen.
Dazu wird eine Kurve gemäß Gleichung \eqref{eq:resonanzkurve} nach der Methode der kleinsten Quadrate an das gemessene Spektrum angepasst.
Am Beispiel der $\mathrm{TM}_{010}~\pi$-Beschleunigermode wurde dies in Abbildung \ref{fig:guetefit} dargestellt.
\begin{figure}[htb]
  \centering
  \input{./plots/guete_fit_pi.tex}
  \caption{Gütefit}
  \label{fig:guetefit}
\end{figure}

Ergebnisse des Fits:
\begin{align}
\kappa &= \num{1.01285 +- 0.00091}\\
Q_0 &= \num{29560 +- 13}\\
\nu_0 &= \SI{499.507 +- 0.001}{MHz}
\end{align}

\section{Vermessung der TM010 Beschleunigermode}
Berechnung des elektrischen Feldes\\
Berechnung der charakteristischen Größen des Resonators: Beschleunigungsspannung, Shuntimpedanz und Laufzeitfaktor

Alle anderen Moden (5/6, 1/2, 1/6) haben verschwindende Felder in der zentralen Zelle und können damit nicht gekoppelt werden.

Längenmessung vom Flansch erklären.


\section{Vermessung von Moden höherer Ordnung (PETRA-III)}
