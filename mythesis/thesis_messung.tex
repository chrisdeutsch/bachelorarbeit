%------------------------------------------------------------------------------
\chapter{Störkörpermessung}
\label{sec:stoerkoerpermessung}
%------------------------------------------------------------------------------
Störkörperkonstante\\
Vorspann (Abstimmen auf Luft, Symmetriesierung des Feldes)\\
Resonanzfrequenz und Güte\\
Berechnung des elektrischen Feldes\\
Berechnung von Beschleunigungsspannung, Shuntimpedanz und Laufzeitfaktor\\
Ergebnisse ...\\



PTFE hat verschwindende magnetische Suszeptibilität [QUELLE] und erlaubt die Verwendung vonweshalb Gleichung \eqref{eq:skm_e_feld_normiert} für dielektrische Kugeln zur Bestimmung der Amplitude des elektrischen Feldes im Resonator genutzt werden kann.

Alle anderen Moden (5/6, 1/2, 1/6) haben verschwindende Felder in der zentralen Zelle und können damit nicht gekoppelt werden.

Messung Güte: Erwärmung des Resonators führt zu einer Verringerung der Leitfähigkeit des Kupfers und somit zu einer Reduzierung der Güte.

Leitfähigkeitänderung $Q$

Fit der Resonanzkurve an die Güte\\
Längenmessung\\

Permittivität Teflon \cite{CRC}(S. 2201) $\epsilon_\mathrm{r} = \num{2.1}$.
Berechnen der Störkörperkonstante \todo{Fehler ist noch nicht richtig!}:
\begin{align}
  \alpha_\mathrm{s} = \SI{2.985 +- 0.092e-17}{\ampere\second\metre\squared\per\volt}
\end{align}

Unterschied Vakuum--Luft \cite{pozar} (S. 308):
\begin{align}
\omega \approx \omega_0 \cdot \frac{3 - \epsilon_\mathrm{r}}{2}
\end{align}
Permittivität Luft (trocken und CO2-frei) \cite{CRC} (S.1093):
\begin{align}
\epsilon_\mathrm{r}^\mathrm{Luft} = \num{1.0005364}
\end{align}

\begin{figure}[ht]
  \centering
  \input{./plots/guete_fit_pi.tex}
  \caption{Gütefit}
  \label{fig:gütefit}
\end{figure}

Ergebnisse des Fits:
\begin{align}
\kappa &= \num{1.01285 +- 0.00091}\\
Q_0 &= \num{29560 +- 13}\\
\nu_0 &= \SI{499.507 +- 0.001}{MHz}
\end{align}