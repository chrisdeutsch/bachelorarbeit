%------------------------------------------------------------------------------
\chapter{Störkörpermessungen}
\label{sec:stoerkoerpermessung}
%------------------------------------------------------------------------------


%------------------------------------------------------------------------------
\section{Vorbereitung der Störkörpermessung}
%------------------------------------------------------------------------------

%------------------------------------------------------------------------------
\subsection{Vorbereitung des Resonators}
\label{sec:vorbereitung_resonator}
%------------------------------------------------------------------------------
Bevor mit der Störkörpermessung nach Abschnitt \ref{sec:messmethodik} begonnen werden kann, müssen noch Einstellungen am Resonator erfolgen.
Zum einen wird die Koppelschleife so gedreht, dass eine nahezu kritische Einkopplung ($\kappa \approx 1$) an die $\mathrm{TM}_{010}\text{-}\pi$-Mode des Resonators realisiert wird.
Außerdem werden beide Abstimmstempel so eingestellt, dass sich eine symmetrische Feldverteilung im Resonator ausbildet.
Schließlich muss die Beschleunigermode auf ihre Sollfrequenz im Vakuum von $\SI{499.67}{MHz}$ abgestimmt werden.
Dabei gilt es zu beachten, dass die Luft im Hohlraum des Resonators zu einer zusätzlichen Verstimmung führt.
Ist der Resonator durch ein Medium der relativen Permittivität~$\varepsilon_\mathrm{r}$ und Permeabilität~$\mu_\mathrm{r}$ ausgefüllt, so kann die Verstimmung der Resonanzfrequenzen durch
\begin{align}
	\nu_0(\mathrm{Med.}) = \frac{\nu_0(\mathrm{Vak.})}{\sqrt{\varepsilon_\mathrm{r} \mu_\mathrm{r}}}
	\label{eq:resonanzfrequenz_medium}
\end{align}
beschrieben werden \cite{pusch}.
Mit der relativen Permittivität trockener Luft~$\varepsilon_\mathrm{r}^\mathrm{Luft} = \num{1.0005364}$ unter Normalbedingungen \cite[S.\ 1093]{CRC}, vernachlässigbarer relativer Permeabilität~($\mu_\mathrm{r} \approx 1$) und der Sollfrequenz im Vakuum folgt die Frequenz der Beschleunigermode
\begin{align}
	\nu_0(\mathrm{Luft}) = \SI{499.54}{MHz}
\end{align}
für den luftgefüllten Resonator.
Da diese Frequenz mit Temperatur und Luftfeuchtigkeit variiert, ist ein exaktes Abstimmen auf diese Frequenz nicht zweckmäßig und es genügt die grobe Einstellung.

%------------------------------------------------------------------------------
\subsection{Bestimmung der Störkörperkonstanten}
%------------------------------------------------------------------------------

Zur Berechnung der Störkörperkonstante~$\alpha_\mathrm{s}$ des kugelförmigen Störkörpers aus PTFE mit einem Durchmesser $D = \SI{20.05 +- 0.05}{mm}$ und der zentrischen Bohrung $d = \SI{1.3 +- 0.05}{mm}$ kann Gleichung \eqref{eq:stoerkoerperkonstante} verwendet werden.
Die relative Permittivität von PTFE beträgt im Mittel $\varepsilon_\mathrm{r} = \num{2.1}$ und variiert mit der Dichte und Kristallinität des Materials \cite[S.\ 2201]{CRC}.
Daher wird für den verwendeten Störkörper $\varepsilon_\mathrm{r} = \num{2.1 +- 0.05}$ angenommen.
Es folgt die Störkörperkonstante
\begin{align}
	\alpha_\mathrm{s} = \SI{2.99 +- 0.11e-17}{\ampere\second\metre\squared\per\volt} \eqcomma
	\label{eq:stoerkoerperkonstante_ptfe}
\end{align}
wobei das Volumen der zentrischen Bohrung beachtet wurde.

Bei der Bestimmung der elektrischen Felder gemäß Gleichung \eqref{eq:skm_e_feld_normiert}, wirkt der Fehler der Störkörperkonstanten systematisch auf das resultierende Feld.
Dieser systematische Einfluss muss insbesondere bei der Integration des Feldes über die Störkörperposition (Bestimmung von Beschleunigungsspannung und Shuntimpedanz) gesondert betrachtet werden.
Um diesen Fehler zu verringern, könnte eine direkte Bestimmung der Störkörperkonstanten in einem Referenzresonator durchgeführt werden.
Darauf wurde jedoch in dieser Arbeit verzichtet, da mit einem relativen Fehler von unter $\SI{4}{\percent}$ eine ausreichende Genauigkeit vorliegt.

%------------------------------------------------------------------------------
\subsection{Bestimmung von Resonanzfrequenz und Güte}
\label{sec:resfreq_guete}
%------------------------------------------------------------------------------
Die Bestimmung von Resonanzfrequenz und Güte der Resonatormoden kann anhand des gemessenen Reflexionsspektrums der jeweiligen Resonanz erfolgen.
Dazu wird eine Kurve gemäß Gleichung \eqref{eq:resonanzkurve} nach der Methode der kleinsten Quadrate an das gemessene Spektrum angepasst.
Dies wurde, am Beispiel der $\mathrm{TM}_{010}\text{-}\pi$-Beschleunigermode von PETRA-III, in Abbildung \ref{fig:guetefit} dargestellt.
\begin{figure}[htb]
  \centering
  \input{./plots/guete_fit_pi.tex}
  \caption[Anpassung der Resonanzkurve an das Reflexionsspektrum der $\mathrm{TM}_{010}~\pi$-Mode von PETRA-III]{Anpassung der Resonanzkurve~\eqref{eq:resonanzkurve} an ein gemessenes Reflexionsspektrum der $\mathrm{TM}_{010}\text{-}\pi$-Mode von PETRA-III. Aus Gründen der Übersicht wurde nur jeder 10.\ Messpunkt des VNA aufgetragen. Die Anpassung liefert die Resonanzfrequenz~$\nu_0 = \SI{499.537}{MHz}$ (Anpassungsfehler vernachlässigbar), Güte~$Q_0 = \num{29560 +- 20}$ und den Koppelfaktor~$\kappa = \num{1.013 +- 0.001}$.}
  \label{fig:guetefit}
\end{figure}
Um eine bessere Abschätzung der Fehler zu erlauben, wurden zu jeder Resonatormode mehrere Reflexionsspektren aufgenommen und an jedes Spektrum eine Anpassung durchgeführt.
Die resultierenden Ergebnisse folgen aus der Bildung des Mittelwerts der angepassten Parameter.
Da die Resonanzfrequenz der Moden mit der Temperatur der Kavität schwankt, wird auf eine Angabe des Fehlers verzichtet.

%------------------------------------------------------------------------------
\section{Vermessung der $\mathrm{TM}_{010}$-Resonatormoden}
\label{sec:tm010_messung}
%------------------------------------------------------------------------------
Die folgenden Abschnitte widmen sich der Auswertung der Störkörpermessungen an den $\mathrm{TM}_{010}$-Resonatormoden von PETRA-III und -IV, welche gemäß Abschnitt \ref{sec:messmethodik} durchgeführt wurden.
Für beide Resonatoren wurden die $\pi,\, 2/3~\pi, \, 1/3~\pi$ und $0$-Moden vermessen.
Die restlichen Moden der siebenzelligen Resonatorkette, die nach den Erläuterungen in Abschnitt \ref{sec:petra_resonator} erwartet werden, konnten nicht vermessen werden.
Dies ist der Fall, da diese Moden ein verschwindendes elektrisches und magnetisches Feld in der mittleren Zelle des Resonators aufweisen (vgl.\ Abb.\ \ref{fig:spektrum_tm010}, \ref{fig:phasenbeziehung}) und daher nicht oder nur schwach mit der Koppelschleife angeregt werden können.
Die Resonatoren wurden mit einer Schrittweite des Störkörpers von \SI{5}{\milli\metre} vermessen, was \num{60} Messpunkten pro Zelle entspricht.
Diese Schrittweite liegt unterhalb des Auflösungsvermögens, welches mit dem Störkörper mit dem Durchmesser von ca.\ \SI{2}{\centi\metre} erreicht werden kann.

\subsection{Auswertung der Messdaten}
Nachdem die Güte~$Q_0$, Resonanzfrequenz~$\nu_0$ und Störkörperkonstante~$\alpha_\mathrm{s}$ bestimmt wurde, kann die Amplitude des elektrischen Feldes (normiert auf die Wurzel der Verlustleistung $P_\mathrm{V}$) gemäß Gleichung \eqref{eq:skm_e_feld_normiert} berechnet werden.
Außerdem kann das effektive elektrische Feld, das ein ultrarelativistisches Teilchen erfährt, welches den Resonator passiert, berechnet und somit der Laufzeitfaktor~$\Lambda$ aus Gleichung \eqref{eq:laufzeitfaktor} bestimmt werden.
Dazu muss neben der harmonischen Zeitabhängigkeit auch die Phasenbeziehung (vgl.\ Abb.\ \ref{fig:phasenbeziehung}) zwischen den einzelnen Zellen beachtet werden.
Darüber hinaus wird die Eintrittsphase des Teilchens in den Resonator so gewählt, dass der Laufzeitfaktor~$\Lambda$ / die effektive Shuntimpedanz~$R_\mathrm{S}^\mathrm{eff}$ maximiert wird.

Dies wurde am Beispiel der $\pi$-Mode des PETRA-III Resonators in Abbildung \ref{fig:bsp_feld_tm010pi_petra3} aufgetragen.
Im Anhang \ref{app:tm010_felder} wurden die Felder aller vermessenen $\mathrm{TM}_{010}$-Resonatormoden beider Resonatoren zusammengestellt.  
\begin{figure}[h]
	\centering
	\input{./plots/PETRA-III/pi.tex}
	\caption[Elektrische Feldverteilung der $\mathrm{TM}_{010}\text{-}\pi$-Beschleunigermode von PETRA-III]{Elektrische Feldverteilung der $\mathrm{TM}_{010}\text{-}\pi$-Beschleunigermode von PETRA-III. Die Position~$z$ ist relativ zum Vakuumflansch angegeben.}
	\label{fig:bsp_feld_tm010pi_petra3}
\end{figure}

Schließlich können die charakteristischen Größen der $\mathrm{TM}_{010}$-Moden bestimmt werden, wobei die folgenden Überlegungen auf dem Inhalt von Abschnitt \ref{sec:resonator_charakteristiken} basieren.
Zunächst erfolgt die Berechnung der longitudinalen Shuntimpedanzen~$R_\mathrm{S}$, indem Gleichung \eqref{eq:beschleunigungsspannung} in \eqref{eq:shuntimpedanz} eingesetzt wird, sodass man
\begin{align}
	R_\mathrm{S} = \frac{1}{2} \left( \int_0^L \frac{|\ve_0(z)|}{\sqrt{P_\mathrm{V}}} \, \mathrm{d}z \right)^2
\end{align}
erhält.
Demnach kann die Shuntimpedanz durch Integration der (normierten) elektrischen Feldamplitude über die Länge~$L$ des Resonators gewonnen werden.
Diese Integration, sowie die hierauf Folgenden, erfolgen dabei numerisch durch die Trapezregel.
Ebenso kann der Laufzeitfaktor~$\Lambda$ als das Verhältnis des Integrals von effektivem Feld zum Integral der Feldamplitude berechnet werden.
Dieser dient der Berechnung der effektiven Shuntimpedanz~$R_\mathrm{S}^\mathrm{eff}$, die nun durch Gleichung \eqref{eq:eff_shuntimpedanz} bestimmt werden kann.
Alternativ kann die effektive Shuntimpedanz durch Integration über das effektive Feld, analog zur Berechnung der Shuntimpedanz, ermittelt werden.

Die Kenngrößen der verschiedenen $\mathrm{TM}_{010}$-Resonatormoden und deren longitudinalen Shuntimpedanzen wurden in Tabelle \ref{tab:shuntimpedanzen_tm010} zusammengestellt.
\begin{table}[htb]
	\begin{subtable}{1\textwidth}
		\centering
		\begin{tabular}{
		c
		S[table-format=3.2]
		S[table-format=5.0(3), table-align-uncertainty = true]
		S[table-format=2.1(3), table-align-uncertainty = true]
		S[table-format=0.3(1), table-align-uncertainty = true]
		S[table-format=3.2(2)e1, table-align-uncertainty = true]
		}
	\toprule
	{$\Delta \varphi$} & {$\nu_0$ / \si{MHz}} & {$Q_0$} & {$R_\mathrm{S}$ / \si{\mega\ohm}} & {$\Lambda$} & {$R_\mathrm{S}^\mathrm{eff}$ / \si{\ohm}} \\
	\midrule
	$\pi$ & 499.67 & 29556+-110 & 43.6+-1.6 & 0.767+-0.002 & 25.65+-0.91e6 \\[0.25em]
	$\frac{2}{3}\pi$ & 501.14 & 31741+-86 & 37.8+-1.4 & 0.048+-0.001 & 85.9+-3.0e3 \\[0.25em]
	$\frac{1}{3}\pi$ & 505.37 & 32707+-118 & 42.3+-1.5 & 0.040+-0.001 & 69.1+-2.4e3 \\[0.25em]
	$0$ & 508.61 & 35999+-66 & 46.2+-1.7 & 0.010+-0.001 & 5.0+-0.5e3 \\
	\bottomrule
\end{tabular}

		\caption{PETRA-III}
	\end{subtable}
	\begin{subtable}{1\textwidth}
		\centering
		\begin{tabular}{
		c
		S[table-format=3.2]
		S[table-format=5.0(3), table-align-uncertainty = true]
		S[table-format=2.1(3), table-align-uncertainty = true]
		S[table-format=0.3(1), table-align-uncertainty = true]
		S[table-format=3.2(2)e1, table-align-uncertainty = true]
		}
	\toprule
	{$\Delta \varphi$} & {$\nu_0$ / \si{MHz}} & {$Q_0$} & {$R_\mathrm{S}$ / \si{\mega\ohm}} & {$\Lambda$} & {$R_\mathrm{S}^\mathrm{eff}$ / \si{\ohm}} \\
	\midrule
	$\pi$ & 499.67 & 28200+-176 & 41.6+-1.5 & 0.767+-0.001 & 24.47+-0.87e6 \\[0.25em]
	$\frac{2}{3}\pi$ & 501.17 & 31356+-218 & 37.5+-1.4 & 0.056+-0.001 & 115.3+-4.1e3 \\[0.25em]
	$\frac{1}{3}\pi$ & 505.43 & 32732+-54 & 42.3+-1.5 & 0.041+-0.001 & 69.8+-2.4e3 \\[0.25em]
	$0$ & 508.61 & 35445+-59 & 45.4+-1.6 & 0.013+-0.001 & 7.7+-0.6e3 \\
	\bottomrule
\end{tabular}

		\caption{PETRA-IV}
	\end{subtable}
	\caption{Longitudinale Shuntimpedanzen der vermessenen $\mathrm{TM}_{010}$-Moden beider PETRA-Resonatoren. Die Resonanzfrequenz~$\nu_0$ wurde gemäß Abschnitt \ref{sec:vorbereitung_resonator} auf die Frequenz des evakuierten Resonators umgerechnet. \todo{Rundungsfehler?}}
	\label{tab:shuntimpedanzen_tm010}
\end{table}

\subsection{Vergleich und Interpretation der Ergebnisse}
Wie zu erwarten, zeigt die $\pi$-Mode beider Resonatoren mit einem Laufzeitfaktor von \SI{76,7}{\percent} die höchste effektive Shuntimpedanz.
Dies ist der Fall, da ein ultrarelativistisches Teilchen zum Durchqueren einer Zelle der Länge \SI{30}{\centi\metre} bei einer treibenden Hochfrequenz \SI{499.67}{MHz} genau eine halbe Periode der Hochfrequenz benötigt.
Durch den Phasensprung von $\pi$ zwischen jeder Zelle bedeutet dies, dass das Teilchen stets ein elektrisches Feld erfährt, welches in dessen Bewegungsrichtung zeigt (vgl.\ Abb.\ in Abschnitt \ref{app:tm010_felder}).
Im Vergleich dazu sind die restlichen $\mathrm{TM}_{010}$-Moden mit effektiven Shuntimpedanzen der Größenordnung von einigen \si{\kilo\ohm} ungeeignet für die effiziente Beschleunigung geladener Teilchen.
Dahingehend soll sich die Diskussion auf die $\pi$-Beschleunigermode beschränken.

Laut Herstellerangaben liegt die Güte der siebenzelligen PETRA-Resonatoren im Bereich von \num{29000} bis \num{36000} bei einem Nominalwert von \num{32800} \cite{desy_petra}.
PETRA-IV unterschreitet somit die Untergrenze der Güte um etwa \num{800} und lediglich PETRA-III fällt in den angegebenen Bereich.
Außerdem unterschreiten beide deutlich die nominale Güte, was möglicherweise darauf zurückzuführen ist, dass beide Hohlräume belüftet waren und sich somit Anlagerungen (Staub, etc.) auf der Kupferoberfläche gebildet haben könnten, die eine Vergrößerung der Verluste des Resonators zur Folge hat.
Im Folgenden wird diese Abweichung auch bei den Shuntimpedanzen auftreten, da diese eine lineare Abhängigkeit von der Güte der Resonatormode aufweist.

Der Hersteller gibt eine nominale (effektive) Shuntimpedanz von \SI{28.1}{\mega\ohm} an \cite{desy_petra}, welche gemäß der vorigen Erläuterungen nicht erreicht werden kann.
Ein Vergleich ist dennoch möglich, wenn der Geometriefaktor $R_\mathrm{S}^\mathrm{eff} / Q_0$ der Beschleunigermode betrachtet wird.
Für beide Resonatoren erhält man den Geometriefaktor der $\pi$-Mode
\begin{align}
	\frac{R_\mathrm{S}^\mathrm{eff}}{Q_0} = \SI{868+-32}{\ohm} \eqcomma
\end{align}
welcher unabhängig von der Verlustleistung des jeweiligen Resonators ist.
Gemäß der Simulation der Kavität durch den Hersteller mit MAFIA\texttrademark\footnote{Programm zur numerischen Simulation elektromagnetischer Felder und Vorgänger des \textit{CST Microwave Studio\textsuperscript{\textregistered}}.} beträgt der Geometriefaktor \SI{856}{\ohm} \cite{desy_petra} und steht somit in guter Übereinstimmung mit den gemessenen Werten.

Wird eine treibende Hochfrequenzleistung~$P_\mathrm{HF}$ bei kritischer Kopplung (die Verlustleistung~$P_\mathrm{V}$ entspricht dann $P_\mathrm{HF}$) des Resonators angenommen, so erhält man die effektive Beschleunigungsspannung gemäß Gleichung \eqref{eq:eff_shuntimpedanz}
\begin{align}
	U_\mathrm{eff} = \sqrt{2 R_\mathrm{S}^\mathrm{eff} P_\mathrm{HF}} \eqdot
\end{align}
Wird beispielsweise zum Betrieb der Resonatoren ein $\SI{200}{\kilo\watt}$-Klystron\footnote{Verstärker für Hochfrequenzsignale} genutzt, dessen Leistung durch ein magisches T (Leistungsteiler auf Hohlleiterbasis) gleichmäßig auf beide Resonatoren aufgeteilt wird, so erhält man effektive Beschleunigungsspannungen
\begin{align}
	U_\mathrm{eff}^\mathrm{P-III} = \SI{2.265 +- 0.041}{\mega\volt} \qquad U_\mathrm{eff}^\mathrm{P-IV} = \SI{2.212 +- 0.040}{\mega\volt} 
\end{align}
unter der Verwendung der gemessenen Shuntimpedanzen.
Der Vergleich mit den aktuell in ELSA verbauten fünfzelligen Resonatoren vom Typ PETRA mit einer effektiven Shuntimpedanz von \SI{15}{\mega\ohm} liefert bei gleicher eingekoppelter Leistung eine effektive Beschleunigungsspannung $U_\mathrm{eff} = \SI{1.73}{\mega\volt}$.
Dies zeigt eine wesentlich effizientere Beschleunigung in den siebenzelligen Resonatoren gegenüber den Fünfzelligen und ist der Hauptgrund für die Konstruktion mehrzelliger Beschleunigungsresonatoren.

Letztlich sei erwähnt, dass bei der Beschleunigung von ultrarelativistischen Teilchen in Kreisbeschleunigern \todo{Overvoltage Factor?}



-andere Moden\\
\todo{Gütezunahme mit der Mode (weniger Stromfluss in den Zwischenwänden)}

\section{Vermessung von Moden höherer Ordnung}
\label{sec:hom_messung}
Schließlich wurde eine Vermessung von Moden höherer Ordnung von PETRA-III durchgeführt.
Die Vermessung und Teile der Auswertung dieser Moden erfolgt analog zu Abschnitt \ref{sec:tm010_messung} und soll daher nicht wiederholt werden.
Die Abweichungen werden im folgenden Abschnitt dargestellt.

\subsection{Auswertung der Messdaten}
Zur Auswertung vermessenen Feldverteilungen musste zunächst die Phasenbeziehung der Felder in den einzelnen Zellen bestimmt werden.
Dazu wurden Simulationen der Eigenmoden eines vereinfachten Modells des siebenzelligen PETRA-Resonators \todo{Schedler} mit \textit{CST Microwave Studio\textsuperscript{\textregistered}} (CST) durchgeführt.
Dadurch das ein vereinfachtes Modell der Kavität genutzt wurde, kam es teilweise zu großen Abweichungen der Resonanzfrequenz zwischen Simulation und Messung.
Darüber hinaus traten Diskrepanzen zwischen simulierten und gemessenen Feldamplituden auf.
Diese Abweichungen sind unter anderem auf die rudimentäre Modellierung der Nasenkegel und der Abstimmstempel zurückzuführen.
Dennoch ist eine Identifizierung der meisten vermessenen Resonatormoden anhand charakteristischer Merkmale der Feldverteilung zu vollziehen.
In Abbildung \ref{fig:cst_sim} wurde exemplarisch die
\begin{figure}[h]
	\centering
	\includegraphics[width=1.0\textwidth]{./figs/TM021-CST/1465_R_cut.png}
	\caption{Simulation des elektrischen Feldes einer der vermessenen $\mathrm{TM}_{021}$-Moden mit der gemessenen Resonanzfrequenz $\nu_0 = \SI{1465.83}{MHz}$. Gezeigt ist der Querschnitt des Modells mit der Strahlenachse in der Ebene. Die Farbe/Größe zeigt Feldstärke. Phase maximalen elektrischen Feldes.}
	\label{fig:cst_sim}
\end{figure}







\begin{figure}[h]
	\centering
	\input{./plots/HOM/1465_R_MHz.tex}
	\caption{Kram2}
	\label{fig:ex_tm021_mode}
\end{figure}	

\begin{table}[htb]
	\centering
	\begin{tabular}{
		c
		c
		S[table-format=4.2]
		S[table-format=5.0(3), table-align-uncertainty = true]
		S[table-format=1.3(3), table-align-uncertainty = true]
		S[table-format=0.3(1), table-align-uncertainty = true]
		S[table-format=3.2(2), table-align-uncertainty = true]
		}
	\toprule
	\multicolumn{2}{c}{Mode} & {$\nu_0$ / \si{MHz}} & {$Q_0$} & {$R_\mathrm{S}$ / \si{\mega\ohm}} & {$\Lambda$} & {$R_\mathrm{S}^\mathrm{eff}$ / \si{\kilo\ohm}} \\
	\midrule
	$\mathrm{TE}_{111}$ & trans. & 702.70 & 11162+-18 & 3.89+-0.14 & 0.289+-0.002 & 326+-11 \\[0.25em]
	$\mathrm{TM}_{011}$ & long. & 730.45 & 13927+-39 & 3.28+-0.12 & 0.118+-0.003 & 45.8+-1.6 \\[0.25em]
	$\mathrm{TM}_{111}$ & trans. & 1047.23 & 28528+-174 & 8.14+-0.29 & 0.017+-0.001 & 2.34+-0.15 \\[0.25em]
	---\textsuperscript{\textasteriskcentered} & long. & 1375.79 & 59762+-416 & 9.47+-0.34 & 0.019+-0.002 & 3.40+-0.40 \\[0.25em]
	$\mathrm{TM}_{021}$\textsuperscript{\textdagger} & long. & 1458.30 & 7552+-51 & 0.566+-0.020 & 0.182+-0.004 & 18.8+-0.7 \\[0.25em]
	$\mathrm{TM}_{021}$\textsuperscript{\textdagger} & long. & 1460.34 & 16043+-63 & 0.729+-0.026 & 0.266+-0.004 & 51.7+-1.8 \\[0.25em]
	$\mathrm{TM}_{021}$ & long. & 1464.96 & 25279+-172 & 2.60+-0.10 & 0.100+-0.002 & 26.1+-1.0 \\[0.25em]
	$\mathrm{TM}_{021}$ & long. & 1465.83 & 15028+-92 & 2.52+-0.09 & 0.269+-0.001 & 183.0+-6.4 \\
	\midrule
	\multicolumn{7}{l}{
		\small{\textsuperscript{\textasteriskcentered}: Nicht Klassifizierbar nach zylindrischen Hohlräumen}
	}\\
	\multicolumn{7}{l}{
		\small{\textsuperscript{\textdagger}: Für Phasenbeziehung wurde worst-case angenommen}
	}\\
	\bottomrule
\end{tabular}

	\caption{PETRA-III Homs}
\end{table}

\subsection{Interpretation der Ergebnisse}

Interpretation:\\
+ Auflösungsvermögen\\
- für HOM Vermessung besser kleinerer Störkörper \textrightarrow Temperaturstabilisierung von Resonator und VNA nötig!\\

+ Phasensprünge (vgl.\ mit realem Fall)
+ Wieso sieht man die Moden mit Feld = 0 ?
+ Was ist mit den Scheiß Nullfeldern?





