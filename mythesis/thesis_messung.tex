%------------------------------------------------------------------------------
\chapter{Störkörpermessungen}
\label{sec:stoerkoerpermessung}
%------------------------------------------------------------------------------
\section{Vorbereitung?}

\subsection{Vorbereitung des Resonators}
Bevor mit der Störkörpermessung nach Abschnitt \ref{sec:messmethodik} begonnen werden kann, müssen noch Einstellungen am Resonator erfolgen.
Zum einen wird die Koppelschleife so gedreht, dass eine nahezu kritische Einkopplung ($\kappa \approx 1$) an die $\mathrm{TM}_{010}~\pi$-Mode des Resonators realisiert wird.
Außerdem werden beide Abstimmstempel so eingestellt, dass sich eine symmetrische Feldverteilung im Resonator ausbildet.
Schließlich muss die Beschleunigermode auf ihre Sollfrequenz im Vakuum von $\SI{499.67}{MHz}$ abgestimmt werden.
Dabei gilt es zu beachten, dass die Luft im Hohlraum des Resonators zu einer zusätzlichen Verstimmung führt.
Ist der Resonator durch ein Medium der relativen Permittivität~$\varepsilon_\mathrm{r}$ und Permeabilität~$\mu_\mathrm{r}$ ausgefüllt, so kann die Verstimmung der Resonanzfrequenzen durch
\begin{align}
	\nu_0(\mathrm{Med.}) = \frac{\nu_0(\mathrm{Vak.})}{\sqrt{\varepsilon_\mathrm{r} \mu_\mathrm{r}}}
\end{align}
beschrieben werden \cite{pusch}.
Mit der relativen Permittivität trockener Luft~$\varepsilon_\mathrm{r}^\mathrm{Luft} = \num{1.0005364}$ unter Normalbedingungen \cite[S.\ 1093]{CRC} und der Sollfrequenz im Vakuum folgt die Frequenz der Beschleunigermode
\begin{align}
	\nu_0(\mathrm{Luft}) = \SI{499.54}{MHz}
\end{align}
für den luftgefüllten Resonator.
Da diese Frequenz mit Temperatur und Luftfeuchtigkeit variiert, ist ein exaktes Abstimmen auf diese Frequenz nicht zweckmäßig und es genügt die grobe Einstellung.

%------------------------------------------------------------------------------
\subsection{Störkörperkonstante}
%------------------------------------------------------------------------------
Zur Berechnung der Störkörperkonstante~$\alpha_\mathrm{s}$ des kugelförmigen Störkörpers aus PTFE kann Gleichung \eqref{eq:stoerkoerperkonstante} verwendet werden.
Die relative Permittivität von PTFE beträgt im Mittel $\varepsilon_\mathrm{r} = \num{2.1}$ und zeigt keine Frequenzabhängigkeit bis Frequenzen der Größenordnung \SI{1}{GHz} \cite[S.\ 2201]{CRC}.
Außerdem variiert die relative Permittivität mit der Dichte und Kristallinität des Materials, weshalb für den verwendeten Werkstoff $\varepsilon_\mathrm{r} = \num{2.1 +- 0.05}$ angenommen werden muss.
Unter Verwendung der Dimensionen des Störkörpers (vgl.\ \ref{sec:aufbau_messstand}) folgt die Störkörperkonstante
\begin{align}
	\alpha_\mathrm{s} = \SI{2.99 +- 0.11e-17}{\ampere\second\metre\squared\per\volt} \eqcomma
	\label{eq:stoerkoerperkonstante_ptfe}
\end{align}
wobei die zentrische Bohrung bei der Berechnung beachtet wurde.

\todo{wirkt als systematisch Fehler}

%------------------------------------------------------------------------------
\subsection{Bestimmung von Resonanzfrequenz und Güte}
\label{sec:resfreq_guete}
%------------------------------------------------------------------------------
Die Bestimmung von Resonanzfrequenz und Güte der Resonatormoden kann anhand des gemessenen Reflexionsspektrums der jeweiligen Resonanz erfolgen.
Dazu wird eine Kurve gemäß Gleichung \eqref{eq:resonanzkurve} nach der Methode der kleinsten Quadrate an das gemessene Spektrum angepasst.
Dies wurde, am Beispiel der $\mathrm{TM}_{010}~\pi$-Beschleunigermode von PETRA-III, in Abbildung \ref{fig:guetefit} dargestellt.
\begin{figure}[htb]
  \centering
  \input{./plots/guete_fit_pi.tex}
  \caption[Anpassung der Resonanzkurve an das Reflexionsspektrum der $\mathrm{TM}_{010}~\pi$-Mode von PETRA-III]{Anpassung der Resonanzkurve~\eqref{eq:resonanzkurve} an ein gemessenes Reflexionsspektrum der $\mathrm{TM}_{010}~\pi$-Mode von PETRA-III. Aus Gründen der Übersicht wurde nur jeder 10.\ Messpunkt des VNA aufgetragen. Die Anpassung liefert die Resonanzfrequenz~$\nu_0 = \SI{499.537 +- 0.001}{MHz}$, Güte~$Q_0 = \num{29560 +- 20}$ und den Koppelfaktor~$\kappa = \num{1.013 +- 0.001}$.}
  \label{fig:guetefit}
\end{figure}
Um eine bessere Abschätzung der Fehler zu erlauben, wurden zu jeder Resonatormode mehrere Reflexionsspektren 



\section{Vermessung der TM010 Beschleunigermode}
Berechnung des elektrischen Feldes\\
Berechnung der charakteristischen Größen des Resonators: Beschleunigungsspannung, Shuntimpedanz und Laufzeitfaktor

Alle anderen Moden (5/6, 1/2, 1/6) haben verschwindende Felder in der zentralen Zelle und können damit nicht gekoppelt werden.

Längenmessung vom Flansch erklären.


\section{Vermessung von Moden höherer Ordnung (PETRA-III)}
