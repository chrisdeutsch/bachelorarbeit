%------------------------------------------------------------------------------
\chapter{Störkörpermessung}
\label{sec:stoerkoerpermessung}
%------------------------------------------------------------------------------
\section{Vorbereitung}
\subsection{Vorbereitung}

Vor der Vermessung der Eigenmoden eines Resonators, wurden die Abstimmstempel so eingestellt, dass sich im Resonator eine symmetrische Feldverteilung aufbaut.
Außerdem muss beachtet werden, dass der Resonator für die Messung nicht evakuiert, sondern durch Luft gefüllt ist.
Dadurch kommt es zu einer Verstimmung der Resonanzfrequenz aufgrund der dielektrischen Eigenschaften von Luft.
Zur Quantifizierung dieses Effekts nutzt man, dass die Resonanzfrequenz~$\nu_0$ einer Resonatormode proportional zur Lichtgeschwindigkeit im Medium~$c$ ist (vgl.~Gl.~\eqref{eq:frequenz_pillbox}) \cite{pusch}.
Da für die Lichtgeschwindigkeit im Medium mit der Permittivität~$\varepsilon$ und der Permeabilität~$\mu$ 
\begin{align}
	c = \frac{1}{\sqrt{\varepsilon \mu}}
\end{align}
gilt, kann die Resonanzfrequenz einer Mode des durch
\begin{align}
	\nu_0(\mathrm{Med.}) = \frac{\nu_0(\mathrm{Vak.})}{\sqrt{\varepsilon_\mathrm{r} \mu_\mathrm{r}}}
\end{align}
Mit der relativen Permittivität der Luft (trocken und CO2-frei) $\varepsilon_\mathrm{r}^\mathrm{Luft} = \num{1.0005364}$ und der Designfrequenz im Vakuum $\SI{499.67}{MHz}$



%------------------------------------------------------------------------------
\subsection{Störkörperkonstante}
%------------------------------------------------------------------------------
Zur Berechnung der Störkörperkonstante~$\alpha_\mathrm{s}$ des kugelförmigen Störkörpers aus PTFE kann Gleichung \eqref{eq:stoerkoerperkonstante} verwendet werden.
Die relative Permittivität von PTFE beträgt im Mittel $\varepsilon_\mathrm{r} = \num{2.1}$ und zeigt keine Frequenzabhängigkeit bis Frequenzen der Größenordnung \SI{1}{GHz} \cite[S.\ 2201]{CRC}.
Außerdem variiert die relative Permittivität mit der Dichte und Kristallinität des Materials, weshalb für den verwendeten Werkstoff $\varepsilon_\mathrm{r} = \num{2.1 +- 0.05}$ angenommen werden muss.
Unter Verwendung der Dimensionen des Störkörpers (vgl.\ \ref{sec:aufbau_messstand}) folgt die Störkörperkonstante
\begin{align}
	\alpha_\mathrm{s} = \SI{2.99 +- 0.11e-17}{\ampere\second\metre\squared\per\volt} \eqcomma
	\label{eq:stoerkoerperkonstante_ptfe}
\end{align}
wobei die zentrische Bohrung bei der Berechnung beachtet wurde.

\todo{wirkt als statistischer Fehler}

%------------------------------------------------------------------------------
\subsection{Bestimmung von Resonanzfrequenz und Güte}
\label{sec:resfreq_guete}
%------------------------------------------------------------------------------
Die Bestimmung von Resonanzfrequenz und Güte der Resonatormoden kann anhand des gemessenen Reflexionsspektrums der jeweiligen Resonanz erfolgen.
Dazu wird eine Kurve gemäß Gleichung \eqref{eq:resonanzkurve} nach der Methode der kleinsten Quadrate an das gemessene Spektrum angepasst.
Dies wurde, am Beispiel der $\mathrm{TM}_{010}~\pi$-Beschleunigermode von PETRA-III, in Abbildung \ref{fig:guetefit} dargestellt.
\begin{figure}[htb]
  \centering
  \input{./plots/guete_fit_pi.tex}
  \caption[Anpassung der Resonanzkurve an das Reflexionsspektrum der $\mathrm{TM}_{010}~\pi$-Mode von PETRA-III]{Anpassung der Resonanzkurve~\eqref{eq:resonanzkurve} an ein gemessenes Reflexionsspektrum der $\mathrm{TM}_{010}~\pi$-Mode von PETRA-III. Aus Gründen der Übersicht wurde nur jeder 10.\ Messpunkt des VNA aufgetragen. Die Anpassung liefert die Resonanzfrequenz~$\nu_0 = \SI{499.507 +- 0.001}{MHz}$, Güte~$Q_0 = \num{29560 +- 20}$ und den Koppelfaktor~$\kappa = \num{1.013 +- 0.001}$.}
  \label{fig:guetefit}
\end{figure}
Um eine bessere Abschätzung der Fehler zu erlauben, wurden zu jeder Resonatormode mehrere Reflexionsspektren 



\section{Vermessung der TM010 Beschleunigermode}
Berechnung des elektrischen Feldes\\
Berechnung der charakteristischen Größen des Resonators: Beschleunigungsspannung, Shuntimpedanz und Laufzeitfaktor

Alle anderen Moden (5/6, 1/2, 1/6) haben verschwindende Felder in der zentralen Zelle und können damit nicht gekoppelt werden.

Längenmessung vom Flansch erklären.


\section{Vermessung von Moden höherer Ordnung (PETRA-III)}
