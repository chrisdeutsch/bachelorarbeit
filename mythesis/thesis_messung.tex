%------------------------------------------------------------------------------
\chapter{Störkörpermessung}
\label{sec:stoerkoerpermessung}
%------------------------------------------------------------------------------

Permittivität Teflon \cite{CRC}(S. 2201) $\epsilon_\mathrm{r} = \num{2.1}$.
Berechnen der Störkörperkonstante \todo{Fehler ist noch nicht richtig!}:
\begin{align}
  \alpha_\mathrm{s} = \SI{2.985 +- 0.092e-17}{\ampere\second\metre\squared\per\volt}
\end{align}

Unterschied Vakuum--Luft \cite{pozar} (S. 308):
\begin{align}
\omega \approx \omega_0 \cdot \frac{3 - \epsilon_\mathrm{r}}{2}
\end{align}
Permittivität Luft (trocken und CO2-frei) \cite{CRC} (S.1093):
\begin{align}
\epsilon_\mathrm{r}^\mathrm{Luft} = \num{1.0005364}
\end{align}


Mit dem Betrag:
\begin{align}
  | \rho(\omega) | = \sqrt{\frac{(\kappa - 1)^2 + Q_0^2 \left( \frac{\omega}{\omega_0}  - \frac{\omega_0}{\omega}\right)^2}{(\kappa + 1)^2 + Q_0^2 \left( \frac{\omega}{\omega_0}  - \frac{\omega_0}{\omega}\right)^2}}
\end{align}
\begin{figure}[ht]
  \centering
  \input{./plots/guete_fit_pi.tex}
  \caption{Gütefit}
  \label{fig:gütefit}
\end{figure}
