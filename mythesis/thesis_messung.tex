%------------------------------------------------------------------------------
\chapter{Störkörpermessungen}
\label{sec:stoerkoerpermessung}
%------------------------------------------------------------------------------


%------------------------------------------------------------------------------
\section{Vorbereitung?}
%------------------------------------------------------------------------------

%------------------------------------------------------------------------------
\subsection{Vorbereitung des Resonators}
%------------------------------------------------------------------------------
Bevor mit der Störkörpermessung nach Abschnitt \ref{sec:messmethodik} begonnen werden kann, müssen noch Einstellungen am Resonator erfolgen.
Zum einen wird die Koppelschleife so gedreht, dass eine nahezu kritische Einkopplung ($\kappa \approx 1$) an die $\mathrm{TM}_{010}~\pi$-Mode des Resonators realisiert wird.
Außerdem werden beide Abstimmstempel so eingestellt, dass sich eine symmetrische Feldverteilung im Resonator ausbildet.
Schließlich muss die Beschleunigermode auf ihre Sollfrequenz im Vakuum von $\SI{499.67}{MHz}$ abgestimmt werden.
Dabei gilt es zu beachten, dass die Luft im Hohlraum des Resonators zu einer zusätzlichen Verstimmung führt.
Ist der Resonator durch ein Medium der relativen Permittivität~$\varepsilon_\mathrm{r}$ und Permeabilität~$\mu_\mathrm{r}$ ausgefüllt, so kann die Verstimmung der Resonanzfrequenzen durch
\begin{align}
	\nu_0(\mathrm{Med.}) = \frac{\nu_0(\mathrm{Vak.})}{\sqrt{\varepsilon_\mathrm{r} \mu_\mathrm{r}}}
\end{align}
beschrieben werden \cite{pusch}.
Mit der relativen Permittivität trockener Luft~$\varepsilon_\mathrm{r}^\mathrm{Luft} = \num{1.0005364}$ unter Normalbedingungen \cite[S.\ 1093]{CRC} und der Sollfrequenz im Vakuum folgt die Frequenz der Beschleunigermode
\begin{align}
	\nu_0(\mathrm{Luft}) = \SI{499.54}{MHz}
\end{align}
für den luftgefüllten Resonator.
Da diese Frequenz mit Temperatur und Luftfeuchtigkeit variiert, ist ein exaktes Abstimmen auf diese Frequenz nicht zweckmäßig und es genügt die grobe Einstellung.

%------------------------------------------------------------------------------
\subsection{Bestimmung der Störkörperkonstanten}
%------------------------------------------------------------------------------

Zur Berechnung der Störkörperkonstante~$\alpha_\mathrm{s}$ des kugelförmigen Störkörpers aus PTFE mit einem Durchmesser $D = \SI{20.05 +- 0.05}{mm}$ und der zentrischen Bohrung $d = \SI{1.3 +- 0.05}{mm}$ kann Gleichung \eqref{eq:stoerkoerperkonstante} verwendet werden.
Die relative Permittivität von PTFE beträgt im Mittel $\varepsilon_\mathrm{r} = \num{2.1}$ und variiert mit der Dichte und Kristallinität des Materials \cite[S.\ 2201]{CRC}.
Daher wird für den verwendeten Störkörper $\varepsilon_\mathrm{r} = \num{2.1 +- 0.05}$ angenommen.
Es folgt die Störkörperkonstante
\begin{align}
	\alpha_\mathrm{s} = \SI{2.99 +- 0.11e-17}{\ampere\second\metre\squared\per\volt} \eqcomma
	\label{eq:stoerkoerperkonstante_ptfe}
\end{align}
wobei das Volumen der zentrischen Bohrung beachtet wurde.

Bei der Bestimmung der elektrischen Felder gemäß Gleichung \eqref{eq:skm_e_feld_normiert}, wirkt der Fehler der Störkörperkonstanten systematisch auf das resultierende Feld.
Dieser systematische Einfluss muss insbesondere bei der Integration des Feldes über die Störkörperposition (Bestimmung von Beschleunigungsspannung und Shuntimpedanz) gesondert betrachtet werden.
Um diesen Fehler zu verringern, könnte eine direkte Bestimmung der Störkörperkonstanten in einem Referenzresonator durchgeführt werden.
Darauf wurde jedoch in dieser Arbeit verzichtet, da mit einem relativen Fehler von unter $\SI{4}{\percent}$ eine ausreichende Genauigkeit vorliegt.

%------------------------------------------------------------------------------
\subsection{Bestimmung von Resonanzfrequenz und Güte}
\label{sec:resfreq_guete}
%------------------------------------------------------------------------------
Die Bestimmung von Resonanzfrequenz und Güte der Resonatormoden kann anhand des gemessenen Reflexionsspektrums der jeweiligen Resonanz erfolgen.
Dazu wird eine Kurve gemäß Gleichung \eqref{eq:resonanzkurve} nach der Methode der kleinsten Quadrate an das gemessene Spektrum angepasst.
Dies wurde, am Beispiel der $\mathrm{TM}_{010}~\pi$-Beschleunigermode von PETRA-III, in Abbildung \ref{fig:guetefit} dargestellt.
\begin{figure}[htb]
  \centering
  \input{./plots/guete_fit_pi.tex}
  \caption[Anpassung der Resonanzkurve an das Reflexionsspektrum der $\mathrm{TM}_{010}~\pi$-Mode von PETRA-III]{Anpassung der Resonanzkurve~\eqref{eq:resonanzkurve} an ein gemessenes Reflexionsspektrum der $\mathrm{TM}_{010}~\pi$-Mode von PETRA-III. Aus Gründen der Übersicht wurde nur jeder 10.\ Messpunkt des VNA aufgetragen. Die Anpassung liefert die Resonanzfrequenz~$\nu_0 = \SI{499.537 +- 0.001}{MHz}$ \todo{$\Delta$}, Güte~$Q_0 = \num{29560 +- 20}$ und den Koppelfaktor~$\kappa = \num{1.013 +- 0.001}$.}
  \label{fig:guetefit}
\end{figure}
Um eine bessere Abschätzung der Fehler zu erlauben, wurden zu jeder Resonatormode mehrere Reflexionsspektren aufgenommen und an jedes Spektrum eine Anpassung durchgeführt.
Die resultierenden Ergebnisse folgen aus der Bildung des Mittelwerts der angepassten Parameter.

\todo{Frequenzgenauigkeit, Leitfähigkeit $\Delta Q_0 \approx \SI{1}{\percent}$}

%------------------------------------------------------------------------------
\section{Vermessung der $\mathrm{TM}_{010}$-Resonatormoden}
\label{sec:tm010_messung}
%------------------------------------------------------------------------------
Die folgenden Abschnitte widmen sich der Auswertung der Störkörpermessungen an den $\mathrm{TM}_{010}$-Resonatormoden von PETRA-III und -IV, welche gemäß Abschnitt \ref{sec:messmethodik} durchgeführt wurden.
Für beide Resonatoren wurden die $\pi,\, 2/3~\pi, \, 1/3~\pi$ und $0$-Moden vermessen.
Die restlichen Moden der siebenzelligen Resonatorkette, die nach den Erläuterungen in Abschnitt \ref{sec:petra_resonator} erwartet werden, konnten nicht vermessen werden.
Dies ist der Fall, da diese Moden ein verschwindendes elektrisches und magnetisches Feld in der mittleren Zelle des Resonators aufweisen (vgl.\ Abb.\ \ref{fig:spektrum_tm010}, \ref{fig:feldverteilung_tm010}) und daher nicht mit der Koppelschleife angeregt werden können.

\subsection{Auswertung der Messdaten}
Nachdem die Güte~$Q_0$, Resonanzfrequenz~$\nu_0$ und Störkörperkonstante~$\alpha_\mathrm{s}$ bestimmt wurde, kann die Amplitude des elektrischen Feldes (normiert auf die Wurzel der Verlustleistung $P_\mathrm{V}$) gemäß Gleichung \eqref{eq:skm_e_feld_normiert} berechnet werden.
Außerdem kann das effektive elektrische Feld, das ein ultrarelativistisches Teilchen erfährt, welches den Resonator passiert, berechnet werden.
Dazu muss neben der harmonischen Zeitabhängigkeit (analog zu Gl.\ \eqref{eq:laufzeitfaktor}) auch der Phasenvorschub $\Delta \varphi$ zwischen den einzelnen Zellen beachtet werden.
Dies wurde am Beispiel der $\pi$-Mode des PETRA-III Resonators in Abbildung \ref{fig:bsp_feld_tm010pi_petra3} aufgetragen.
Im Anhang \ref{app:tm010_felder} wurden die Felder aller vermessenen $\mathrm{TM}_{010}$-Resonatormoden beider Resonatoren zusammengestellt.  
\begin{figure}[h]
	\centering
	\input{./plots/PETRA-III/pi.tex}
	\caption[Elektrische Feldverteilung der $\mathrm{TM}_{010}~\pi$-Beschleunigermode von PETRA-III]{Elektrische Feldverteilung der $\mathrm{TM}_{010}~\pi$-Beschleunigermode von PETRA-III. Die Position~$z$ ist relativ zum Vakuumflansch angegeben.}
	\label{fig:bsp_feld_tm010pi_petra3}
\end{figure}

\begin{table}
	\begin{subtable}{1\textwidth}
		\centering
		\begin{tabular}{
		c
		S[table-format=3.2]
		S[table-format=5.0(3), table-align-uncertainty = true]
		S[table-format=2.1(3), table-align-uncertainty = true]
		S[table-format=0.3(1), table-align-uncertainty = true]
		S[table-format=3.2(2)e1, table-align-uncertainty = true]
		}
	\toprule
	{$\Delta \varphi$} & {$\nu_0$ / \si{MHz}} & {$Q_0$} & {$R_\mathrm{S}$ / \si{\mega\ohm}} & {$\Lambda$} & {$R_\mathrm{S}^\mathrm{eff}$ / \si{\ohm}} \\
	\midrule
	$\pi$ & 499.67 & 29556+-110 & 43.6+-1.6 & 0.767+-0.002 & 25.65+-0.91e6 \\[0.25em]
	$\frac{2}{3}\pi$ & 501.14 & 31741+-86 & 37.8+-1.4 & 0.048+-0.001 & 85.9+-3.0e3 \\[0.25em]
	$\frac{1}{3}\pi$ & 505.37 & 32707+-118 & 42.3+-1.5 & 0.040+-0.001 & 69.1+-2.4e3 \\[0.25em]
	$0$ & 508.61 & 35999+-66 & 46.2+-1.7 & 0.010+-0.001 & 5.0+-0.5e3 \\
	\bottomrule
\end{tabular}

		\caption{PETRA3}
	\end{subtable}
	\begin{subtable}{1\textwidth}
		\centering
		\begin{tabular}{
		c
		S[table-format=3.2]
		S[table-format=5.0(3), table-align-uncertainty = true]
		S[table-format=2.1(3), table-align-uncertainty = true]
		S[table-format=0.3(1), table-align-uncertainty = true]
		S[table-format=3.2(2)e1, table-align-uncertainty = true]
		}
	\toprule
	{$\Delta \varphi$} & {$\nu_0$ / \si{MHz}} & {$Q_0$} & {$R_\mathrm{S}$ / \si{\mega\ohm}} & {$\Lambda$} & {$R_\mathrm{S}^\mathrm{eff}$ / \si{\ohm}} \\
	\midrule
	$\pi$ & 499.67 & 28200+-176 & 41.6+-1.5 & 0.767+-0.001 & 24.47+-0.87e6 \\[0.25em]
	$\frac{2}{3}\pi$ & 501.17 & 31356+-218 & 37.5+-1.4 & 0.056+-0.001 & 115.3+-4.1e3 \\[0.25em]
	$\frac{1}{3}\pi$ & 505.43 & 32732+-54 & 42.3+-1.5 & 0.041+-0.001 & 69.8+-2.4e3 \\[0.25em]
	$0$ & 508.61 & 35445+-59 & 45.4+-1.6 & 0.013+-0.001 & 7.7+-0.6e3 \\
	\bottomrule
\end{tabular}

		\caption{PETRA4}
	\end{subtable}
	\caption{longitudinale Shuntimpedanzen von PETRA-III \todo{Klammer um Shuntimpedanz mit Fehler}}
\end{table}

\todo{Shuntimpedanzkram (alles was in Python passiert), Gütezunahme mit der Mode (weniger Stromfluss in den Zwischenwänden)}


\subsection{Interpretation}
Berechnung der charakteristischen Größen des Resonators: Beschleunigungsspannung, Shuntimpedanz und Laufzeitfaktor



\section{Vermessung von Moden höherer Ordnung (PETRA-III)}
\label{sec:hom_messung}
Außerdem wurden einige Moden höherer Ordnung für den Resonator PETRA-III vermessen.



