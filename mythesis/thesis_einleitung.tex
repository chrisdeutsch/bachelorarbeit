%==============================================================================
\chapter{Einleitung}
\label{sec:einleitung}
%==============================================================================

\section{ELSA}
Die Elektronen-Stretcher-Anlage ELSA am Physikalischen Institut der Universität Bonn ist eine dreistufige Beschleunigeranlage für Elektronen.
Zur Beschleunigung der Elektronen und Kompensation des Energieverlustes aufgrund von Synchrotronstrahlung werden hochfrequente elektrische Wechselfelder genutzt, weshalb ein Betrieb mit einem in Teilchenpakete aufgeteilten Elektronenstrahl (engl.\ \textit{bunched beam}) erfolgen muss.
In der ersten Stufe der Anlage werden die Elektronen durch den Linearbeschleuniger LINAC2 auf eine Energie von \SI{26}{MeV} beschleunigt, wobei wahlweise unpolarisierte oder spinpolarisierte Elektronen genutzt werden können.
Anschließend wird der Elektronenstrahl in ein Booster-Synchrotron injiziert, welches die Elektronenpakete auf Energien von bis zu \SI{1.2}{GeV} beschleunigt.
In der letzten Stufe folgt die Injektion in den Stretcherring, in dem die Beschleunigung auf finale Energien von bis zu \SI{3.2}{GeV} und internen Strahlströmen von typischerweise \SI{20}{mA} vollzogen wird.
Schließlich können die (spinpolarisierten) Elektronen einem von zwei Hadronenphysikexperimenten bereitgestellt werden.


\section{Erhöhung des internen Strahlstroms an ELSA}
Um den Hadronenphysikexperimenten höhere Strahlintensitäten bereitstellen zu können, ist eine Erhöhung des internen Strahlstroms auf \SI{200}{mA} bei der maximalen Energie von \SI{3.2}{GeV} geplant.
Derzeitig wird der maximale Strahlstrom durch die begrenzte Hochfrequenzleistung des Stretcherrings limitiert \cite{schedler}, welche durch ein Klystron\footnote{Verstärker für Hochfrequenzsignale} und zwei fünfzellige Hohlraumresonatoren vom Typ PETRA erbracht wird.
Eine Erweiterung des Hochfrequenzsystems um ein weiteres Klystron und zwei zusätzliche siebenzellige PETRA-Hohlraumresonatoren (detaillierte Beschreibung des Resonators folgt in Abschnitt \ref{sec:petra_resonator}), soll schließlich eine Erhöhung des internen Strahlstroms auf \SI{200}{mA} ermöglichen.

\section{Zielsetzung}
Im Rahmen dieser Erweiterung und der vorliegenden Arbeit sollen die elektrischen Felder in den siebenzelligen PETRA-Resonatoren charakterisiert werden.
Dazu kann die Methode der resonanten Störkörpermessung, welche in Abschnitt \ref{sec:resonante_stoerkoerpermessung} eingeführt wird, genutzt werden.
Ziel dieser Arbeit ist die Bestimmung des elektrischen Feldes verschiedener Moden der Resonatoren und die nachfolgende Bestimmung ihrer Shuntimpedanzen\footnote{Maß für die Beschleunigungseffizienz geladener Teilchen einer Resonatormode}.
Insbesondere wird dabei die Fundamentalmode\footnote{Resonatormode mit der niedrigsten Resonanzfrequenz}, die der Beschleunigung der ultrarelativistischen Elektronen im Stretcherring dient, untersucht.
Darüber hinaus werden noch einige Moden höherer Ordnung\footnote{Resonatormoden, deren Resonanzfrequenzen oberhalb der Fun\-da\-men\-tal\-mode liegen} betrachtet, welche durch die im Beschleuniger umlaufenden Elektronenpakete angeregt werden können.
%Die Rückwirkung solcher Moden auf die Bunche kann zur Ausbildung von sog.\ Multi-Bunch-Instabilitäten führen, welche durch aktive oder passive Methoden gedämpft werden müssen und daher ebenfalls von Interesse sind.
Die Rückwirkung solcher Moden auf die Bunche kann zur Ausbildung von sog.\ Multi-Bunch-Instabilitäten führen, welche sich durch eine Anregung kohärenter Schwingungen des Strahls auszeichnen.
Dies resultiert in einem Anstieg der Strahlschwingungsamplitude und kann zu Strahlverlust führen, weshalb aktive oder passive Methoden zur Dämpfung der Strahlschwingungen benutzt werden.
Insbesondere liegt der Schwerpunkt auf den $\mathrm{TM}_{021}$-Moden, die gemäß Simulationen in \cite{schedler} Shuntimpedanzen der Größenordnung der Fundamentalmode aufweisen und somit eine signifikante Quelle für die Anregung von Multi-Bunch-Instabilitäten sein können.
