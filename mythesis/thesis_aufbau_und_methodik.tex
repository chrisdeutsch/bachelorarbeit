%==============================================================================
\chapter{Aufbau und Methodik}
\label{sec:aufbau_und_methodik}
%==============================================================================


%------------------------------------------------------------------------------
\section{Der PETRA-Resonator}
%------------------------------------------------------------------------------
\begin{figure}[htb]
  \centering
  \includegraphics[width=0.8\textwidth]{./figs/cavity/cavity.pdf}
  \caption{Schematischer Darstellung des Querschnitts eines siebenzelligen PETRA-Resonators mit Koppelschleife und zwei Abstimmstempeln. Der Abstand zwei benachbarter Zellenmitten beträgt \SI{30}{\centi\metre}.}
  \label{fig:petra_cavity}
\end{figure}
Im Rahmen dieser Arbeit wird das elektrische Feld von zwei siebenzelligen Beschleunigungsresonatoren vom Typ PETRA \cite{desy_petra} vermessen, welche in Kooperation von DESY und Balzers Hochvakuum GmbH (1982/83) hergestellt wurden.
Der Resonator ist ausgelegt auf die Beschleunigung von ultrarelativistischen Elektronen und Positronen durch die $\mathrm{TM}_{010}$-Mode bei einer Frequenz von \SI{499.67}{MHz}.
Der Querschnitt dieser Kavität ist in Abbildung \ref{fig:petra_cavity} schematisch dargestellt.
Die einzelnen Zellen werden durch leitende Kupferwände voneinander getrennt, wobei deren spezielle Form der Maximierung des Laufzeitsfaktors \eqref{eq:laufzeitfaktor} dient.
Jede der Zellen des PETRA-Resonators stellt einen Hohlraumresonator dar, der durch vier Koppelschlitze in der trennenden Kupferwand an die jeweiligen Nachbarzellen gekoppelt ist.
Dies führt zur Aufspaltung jeder Mode der Einzelzelle in $N$ Moden der Resonatorkette\footnote{Dies ist analog zu gekoppelten mechanischen Pendeln, welche ebenfalls verschiedene Schwingungsmoden aufweisen.}, wobei $N = 7$ die Anzahl der Zellen ist.
Die $N$ Moden der Resonatorkette unterscheiden sich neben der Resonanzfrequenz $\omega_0$ auch im Phasenvorschub $\Delta \varphi$ zwei benachbarter Zellen.
Die Benennung der Moden der Resonatorkette erfolgt nach ihrem Phasenvorschub $\Delta \varphi$ (z.B.\ $\pi$-Mode für $\Delta \varphi = \pi$) und die für den siebenzelligen PETRA-Resonator möglichen Moden sind:
\begin{align}
  \Delta \varphi = 0,\; \frac{1}{6} \pi,\; \frac{1}{3} \pi,\; \frac{1}{2} \pi,\; \frac{2}{3} \pi,\; \frac{5}{6} \pi,\; \pi
\end{align}
In der mittleren Zelle des PETRA-Resonators befindet sich die Koppelschleife \cite{desy_schleife}, deren Geometrie auf den Resonator angepasst ist um eine kritische Kopplung an die $\mathrm{TM}_{010}$-Beschleuniger\-mode zu ermöglichen.
Eine Anpassung des Koppelfaktors kann durch Drehen der Schleife erreicht werden, da dies die effektive Fläche der Leiterschleife senkrecht zum Magnetfeld der betrachteten Resonatormode verändert.
In den Zellen~2 und 6 sind Abstimmstempel \cite{desy_stempel} angebracht die durch Schrittmotoren in bzw.\ aus der Kavität gefahren werden können.
Das Verstellen der Stempel führt zu einer Änderung der Geometrie des Resonators und folglich zu einer Verschiebung der Resonanzfrequenz.
Dies kann zum Stimmen der Resonanzfrequenz einer Mode in einem beschränkten Frequenzbereich genutzt werden.
Die zwei weiteren Vakuumflansche in den Zellen~1 und 7 dienen dem Anschluss von Vakuumpumpen.


Die in dieser Arbeit vermessenen Resonatoren seien im Folgenden mit PETRA-III und -IV bezeichnet\footnote{Die fünfzelligen Resonatoren PETRA-I und -II sind im Stretcherring von ELSA verbaut.}. Der Resonator PETRA-IV wurde durch eine Gasentladung gereinigt.

\todo{Länge Flansch-Flansch}

%------------------------------------------------------------------------------
\section{Aufbau des Störkörpermessstandes}
%------------------------------------------------------------------------------
\begin{sidewaysfigure}[p]
	\centering
	\includegraphics[width=1.0\textheight]{./figs/cavity/messaufbau.pdf}
	\caption{Schematischer Aufbau des Störkörpermessstandes für einen siebenzelligen PETRA-Resonator mit Abstimmstemplen und Einkoppelschleife (a), Montageträger (b), Gestell aus einem Aluminiumprofilsystem (c), Kreuzblenden (d), Polytetrafluorethylen-Störkörper (e), Spindel mit Schrittmotor (f) und Zugfeder (g).}
	\label{fig:stoerkoerpermessstand}
\end{sidewaysfigure}
Zur Bestimmung des elektrischen Feldes auf der Strachachse des Resonators wird die in Abschnitt \ref{sec:resonante_stoerkoerpermessung} eingeführte resonante Störkörpermessung ausgenutzt.
Dazu wird ein kugelförmiger Störkörper mit dem Durchmesser $D = \SI{20.05 +- 0.02}{mm}$ durch eine zentrische Bohrung des Durchmessers $d = \SI{1.3 +- 0.05}{mm}$ auf einem reißfesten Kunststofffaden befestigt.
Wegen der verschwindenden magnetischen Suszeptibilität wurde der Störkörper aus Polytetrafluorethylen (Teflon\textsuperscript{\textregistered}) gefertigt, was die Verwendung von Gleichung \eqref{eq:skm_e_feld_normiert} für dielektrische Kugeln zur Bestimmung des Amplitude des elektrischen Feldes im Resonator erlaubt.
Um diesen Störkörper auf der Strahlachse durch den Resonator bewegen zu können, wurde der Störkörpermessstand aus Abbildung \ref{fig:stoerkoerpermessstand} aufgebaut.
Dieser besteht aus dem siebenzelligen PETRA-Resonator, welcher fest auf einem Montageträger befestigt wurde.
Ein Gestell aus einem Aluminiumprofilsystem, welches fest mit dem Träger verbunden wurde, dient der Befestigung der weiteren Komponenten des Messstandes.
Über ein System von Umlenkrollen kann der Kunststofffaden durch die Kavität geführt werden und schließlich beidseitig fest mit einer Aluminiumspindel verbunden werden, die wiederum auf der Welle eines Schrittmotors befestigt ist.
Die Rückführung des Fadens auf die Spindel führt dazu, dass unabhängig von der Spannung des Fadens das Drehmoment auf der Welle des Schrittmotors kompensiert wird.
Darüber hinaus verhindert die feste Verbindung des Fadens mit der Spindel ein longitudinales Verrutschen des Fadens in seiner Führung, was insbesondere für eine präzise Ortsbestimmung des Störkörpers relevant ist.

Durch das Auf- beziehungsweise Abrollen des Fadens auf der Spindel, kann der Störkörper entlang der Strahlachse verschoben werden.
Die Drehung der Spendel erfolgt dabei durch den 2-Phasen Schrittmotor \textit{MDrive\textsuperscript{\textregistered} 23 Plus Motion Control} des Herstellers \textit{Schneider Electric} \cite{motor_datasheet}.
Der Motor besitzt eine integrierte Steuereinheit, welche die direkte Ansteuerung über eine serielle Schnittstelle ermöglicht.
Ein voller Schritt des Motors entspricht einem Drehwinkel der Welle von \SI{1.8}{\degree}, was \num{200} Schritten pro Umdrehung entspricht.
Durch eine spezielle Ansteuerungstechnik, welche in der Steuereinheit implementiert wurde, wird ein Schritt zusätzlich in \num{256} sogenannte \textit{Microsteps} eingeteilt.
Die typische Stellgenaugikeit für Schrittmotoren (innerhalb der Betriebsparameter) liegt bei \SI{5}{\percent} eines Schrittes und entspricht einem Winkelfehler von \SI{0.09}{\degree}.





\todo{Schrittverlust}

An die Koppelschleife ist der vektorielle Netzwerkanalysator \textit{ZVC 1127.8600.62} des Herstellers \textit{Rhode \& Schwarz} angeschlossen.

Die Steuerung der Komponenten erfolgt durch den Mikrocomputer Raspberry Pi mithilfe der Python3

\todo{Einfluss des Fadens?, Durchhängen des Fadens}






%------------------------------------------------------------------------------
\section{Messmethodik}
%------------------------------------------------------------------------------

\subsection{Temperatur}
Volumenänderung $\nu$, Leitfähigkeitänderung $Q$

Fit der Resonanzkurve an die Güte\\
Resonanzfrequenz aus dem Minimum der Resonanzkurve\\
Längenmessung\\
Checkmode
