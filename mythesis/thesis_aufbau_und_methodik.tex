%==============================================================================
\chapter{Aufbau und Methodik}
\label{sec:aufbau_und_methodik}
%==============================================================================


%------------------------------------------------------------------------------
\section{Der PETRA-Resonator}
%------------------------------------------------------------------------------
\begin{figure}[htb]
  \centering
  \includegraphics[width=0.8\textwidth]{./figs/cavity/cavity.pdf}
  \caption{Schematischer Darstellung des Querschnitts eines siebenzelligen PETRA-Resonators mit Koppelschleife und zwei Abstimmstempeln. Der Abstand zwei benachbarter Zellenmitten beträgt \SI{30}{\centi\metre} und die Gesamtlänge des Resonators \SI{222}{\centi\metre}.}
  \label{fig:petra_cavity}
\end{figure}
Im Rahmen dieser Arbeit wird das elektrische Feld von zwei siebenzelligen Beschleunigungsresonatoren vom Typ PETRA \cite{desy_petra} vermessen, welche in Kooperation von DESY und Balzers Hochvakuum GmbH (1982/83) hergestellt wurden.
Der Resonator ist ausgelegt auf die Beschleunigung von ultrarelativistischen Elektronen und Positronen durch die $\mathrm{TM}_{010}$-Mode bei einer Frequenz von \SI{499.67}{MHz}.
Der Querschnitt dieser Kavität ist in Abbildung \ref{fig:petra_cavity} schematisch dargestellt.
Die einzelnen Zellen werden durch leitende Kupferwände voneinander getrennt, wobei deren spezielle Form der Maximierung des Laufzeitsfaktors \eqref{eq:laufzeitfaktor} dient.
Jede der Zellen des PETRA-Resonators stellt einen Hohlraumresonator dar, der durch vier Koppelschlitze in der trennenden Kupferwand an die jeweiligen Nachbarzellen gekoppelt ist.
Dies führt zur Aufspaltung jeder Mode der Einzelzelle in $N$ Moden der Resonatorkette\footnote{Dies ist analog zu gekoppelten mechanischen Pendeln, welche ebenfalls verschiedene Schwingungsmoden aufweisen.}, wobei $N = 7$ die Anzahl der Zellen ist.
Die $N$ Moden der Resonatorkette unterscheiden sich neben der Resonanzfrequenz $\omega_0$ auch in der Phasendifferenz $\Delta \varphi$ zwei benachbarter Zellen.
Die Benennung der Moden der Resonatorkette erfolgt nach dem Phasenvorschub pro Zelle $\Delta \varphi$ (z.B.\ $\pi$-Mode für $\Delta \varphi = \pi$) und die für den siebenzelligen PETRA-Resonator möglichen Moden sind:
\begin{align}
  \Delta \varphi = 0,\; \frac{1}{6} \pi,\; \frac{1}{3} \pi,\; \frac{1}{2} \pi,\; \frac{2}{3} \pi,\; \frac{5}{6} \pi,\; \pi
\end{align}
In der mittleren Zelle des PETRA-Resonators befindet sich die Koppelschleife \cite{desy_schleife}, deren Geometrie auf den Resonator angepasst ist um eine kritische Kopplung an die $\mathrm{TM}_{010}$-Beschleuniger\-mode zu ermöglichen.
Eine Anpassung des Koppelfaktors kann durch Drehen der Schleife erreicht werden, da dies die effektive Fläche der Leiterschleife senkrecht zum Magnetfeld der betrachteten Resonatormode verändert.
In den Zellen~2 und 6 sind Abstimmstempel \cite{desy_stempel} angebracht die durch Schrittmotoren in bzw.\ aus der Kavität gefahren werden können.
Das Verstellen der Stempel führt zu einer Änderung der Geometrie des Resonators und folglich zu einer Verschiebung der Resonanzfrequenz.
Dies kann zum Stimmen der Resonanzfrequenz einer Mode in einem beschränkten Frequenzbereich genutzt werden.
Die zwei weiteren Vakuumflansche in den Zellen~1 und 7 dienen dem Anschluss von Vakuumpumpen.


Die in dieser Arbeit vermessenen Resonatoren seien im Folgenden mit PETRA-III und -IV bezeichnet\footnote{Die fünfzelligen Resonatoren PETRA-I und -II sind im Stretcherring von ELSA verbaut.}. Der Resonator PETRA-IV wurde durch eine Gasentladung gereinigt.
\todo{Kupfer}

%------------------------------------------------------------------------------
\section{Aufbau des Störkörpermessstandes}
%------------------------------------------------------------------------------
\begin{sidewaysfigure}[p]
	\centering
	\includegraphics[width=1.0\textheight]{./figs/cavity/messaufbau.pdf}
	\caption{Schematischer Aufbau des Störkörpermessstandes für einen siebenzelligen PETRA-Resonator mit Abstimmstemplen und Einkoppelschleife (a), Montageträger (b), Gestell aus einem Aluminiumprofilsystem (c), Kreuzblenden (d), Polytetrafluorethylen-Störkörper (e), Spindel mit Schrittmotor (f) und Zugfeder (g).}
	\label{fig:stoerkoerpermessstand}
\end{sidewaysfigure}
Zur Bestimmung des elektrischen Feldes auf der Strahlachse des Resonators wird die in Abschnitt \ref{sec:resonante_stoerkoerpermessung} eingeführte resonante Störkörpermessung genutzt.
Dazu wird ein kugelförmiger Störkörper aus Polytetrafluorethylen (PTFE) mit einem Durchmesser $D = \SI{20.05 +- 0.02}{mm}$ verwendet, welcher durch eine zentrische Bohrung $d = \SI{1.3 +- 0.05}{mm}$ auf einem Kunststofffaden befestigt werden kann.

Um den Störkörper auf der Strahlachse durch den Resonator bewegen zu können, wurde der Störkörpermessstand aus Abbildung \ref{fig:stoerkoerpermessstand} aufgebaut.
Über ein Gestell aus einem Aluminiumprofilsystem und einem System von Umlenkrollen, welche fest mit dem siebenzelligen PETRA-Resonator verbunden sind, kann ein Kunststofffaden, an dem der Störkörper befestigt ist, durch den Resonator gespannt werden.
Dieser Faden wird dann beidseitig fest mit einer Spindel verbunden, die wiederum auf der Welle eines Schrittmotors befestigt ist.
Die Rückführung des Fadens auf die Spindel führt dazu, dass unabhängig von der Spannung des Fadens das Drehmoment auf der Welle des Schrittmotors kompensiert wird.
Dadurch kann der Faden so stark gespannt werden, dass das Durchhängen des Störkörpers am Faden vernachlässigt werden kann, ohne dabei einen Schrittverlust durch ein überlastendes Drehmoment am Motor zu riskieren.
Darüber hinaus verhindert die feste Verbindung des Fadens mit der Spindel ein longitudinales Verrutschen des Fadens in seiner Führung, was insbesondere für eine präzise Ortsbestimmung des Störkörpers wichtig ist.
Zusätzlich wurde an den Vakuumflanschen des Resonators Kreuzblenden installiert um einerseits ein Eindringen von Staub zu vermindern. Andererseits dient die Blende als Referenzpunkt zur Zentrierung des Störkörpers auf der Strahlachse.

Durch das Auf- beziehungsweise Abrollen des Fadens auf der Spindel, kann der Störkörper entlang der Strahlachse verschoben werden.
Die Drehung der Spindel erfolgt dabei durch den 2-Phasen Schrittmotor \textit{MDrive\textsuperscript{\textregistered} 23 Plus Motion Control} des Herstellers \textit{Schneider Electric} \cite{motor_datasheet}.
Der Motor besitzt eine integrierte Steuereinheit, welche die direkte Ansteuerung über eine serielle Schnittstelle ermöglicht.
Ein voller Schritt des Motors entspricht einem Drehwinkel der Welle von \SI{1.8}{\degree} oder \num{200} Schritten pro Umdrehung.
Durch eine spezielle Ansteuerungstechnik, welche in der Steuereinheit implementiert wurde, wird ein Schritt zusätzlich in \num{256} sogenannte \textit{Microsteps} eingeteilt.
Die Ortsinformation des Störkörpers entlang der Strahlachse kann dann aus der Anzahl der erfolgten Microsteps gewonnen werden.
Aus der Geometrie der Spindel folgt für die zugehörige Proportionalitätskonstante ca.\ \SI{0.006}{\milli\metre\per\micro\step}.
Die Stellgenauigkeit von Schrittmotoren liegt typischerweise bei \SI{5}{\percent} eines vollen Schrittes \cite{gecko}, was einem Winkelfehler von \SI{0.09}{\degree} oder einem Fehler in der Störkörperposition von weniger als \SI{0.1}{\milli\metre} entspricht.

Zur Bestimmung der Resonanzfrequenz und Güte des Resonators dient der vektorielle Netzwerkanalysator \textit{Rohde \& Schwarz: ZVC 1127.8600.62}, welcher über ein \SI{50}{\ohm}-Koaxialkabel (RG~214/U) mit der Koppelschleife verbunden wird.
Die Verbindung erfolgt dabei über einen Spezialadapter vom Typ-N Stecker des Kabels auf die $6\tfrac{1}{8}$-Zoll \SI{50}{\ohm}-Koaxialverbindung der Koppelschleife.
Der VNA erzeugt ein sinusförmiges Signal variabler Frequenz und einer Leistung von \SI{6}{dBm}, welches an der Koppelschleife teilweise reflektiert wird.
Ein Richtkoppler im VNA trennt das reflektierte vom ausgehenden Signal und ermöglicht eine Bestimmung des komplexen Reflexionsfaktors durch einen Vergleich von Phase und Leistung beider Signale.
Um ein von der Dämpfung und Laufzeit im Koaxialkabel unabhängiges Ergebnis zu erhalten, kann eine Kalibration durchgeführt werden, indem das Ende der Messleitung durch einen offenen, kurzgeschlossenen und \SI{50}{\ohm}-Abschluss terminiert wird.
Der verwendete VNA ermöglicht die Messung des Reflexionsfaktors an bis zu \num{2001} gleichverteilten Messpunkten auf einem vorgegebenen Frequenzintervall.
Die Eingabe und Ausgabe erfolgt auch hier durch eine serielle Schnittstelle.
\todo{AVG, Marker etc.?}

Zur Realisierung einer automatisierten Störkörpermessung und der Koordination von Schrittmotor und Netzwerkanalysator dient der Mikrocomputer \textit{Raspberry Pi} (RPi), der mit den Komponenten des Messaufbaus über serielle Schnittstellen kommunizieren kann.
In der Programmiersprache \textit{Python} wurde dazu die serielle Kommunikation mit Motor und VNA, sowie die Steuerung des Messablaufs implementiert.
Der genaue Ablauf einer Störkörpermessung wird im nächsten Abschnitt erläutert.


%------------------------------------------------------------------------------
\section{Messmethodik}
%------------------------------------------------------------------------------

%------------------------------------------------------------------------------
\subsection{Temperaturabhängigkeit der Resonanzfrequenz}
%------------------------------------------------------------------------------
Für die resonante Störkörpermessung gilt es die Verschiebung der Resonanzfrequenz $\Delta \nu$ durch den Störkörper an verschiedenen Positionen im Resonator zu messen.
Dabei muss man beachten, dass es neben der Frequenzverschiebung durch den Störkörper auch zu einer Verstimmung aufgrund von Temperaturänderungen des Resonators kommt.
Kommt es beispielsweise zu einer Erwärmung des Resonators, so führt diese zu einer Ausdehnung des Kupfers und folglich zu einer Vergrößerung des Resonatorvolumens.
Diese Geometrieänderung des Hohlraums führt zu einer Reduzierung der Resonanzfrequenz des Resonators.
Analog führt ein Abkühlen des Resonators zur Kontraktion des Resonators, die wiederum eine Erhöhung der Resonanzfrequenz zur Folge hat.

Für die vermessenen PETRA-Resonatoren liegt die Verstimmung durch Temperaturänderung bei ungefähr \SI{8}{\kilo\hertz\per\celsius}.
Diese Resonanzfrequenzverschiebung liegt in der gleichen Größenordnung wie die Verschiebung durch den Störkörper, weshalb eine Korrektur dieses Effektes unabdingbar ist.
Eine Möglichkeit ist die präzise Temperaturregulierung der Kavität durch einen thermisch stabilisierten Messraum oder durch den Kühlkreislauf des Resonators.
Für die Messungen im Rahmen dieser Arbeit wurde jedoch eine andere Methode genutzt, um temperaturunabhängige Ergebnisse zu erhalten, welche im Folgenden näher erläutert werden soll.






Leitfähigkeitänderung $Q$

Fit der Resonanzkurve an die Güte\\
Resonanzfrequenz aus dem Minimum der Resonanzkurve\\
Längenmessung\\
Checkmode\\
Einschwingzeit
