%==============================================================================
\chapter{Aufbau und Methodik}
\label{sec:aufbau_und_methodik}
%==============================================================================


%------------------------------------------------------------------------------
\section{Der PETRA-Resonator}
%------------------------------------------------------------------------------


%------------------------------------------------------------------------------
\subsection{Kopplung mehrerer Resonatoren}
%------------------------------------------------------------------------------


%------------------------------------------------------------------------------
\section{Aufbau des Störkörpermessstandes}
%------------------------------------------------------------------------------
\begin{sidewaysfigure}[p]
	\centering
	\includegraphics[width=1.0\textheight]{./figs/cavity/messaufbau.pdf}
	\caption{Schematischer Aufbau des Störkörpermessstandes mit einem siebenzelligen PETRA-Resonator (a),  Koppelschleife (b), Abstimmstempeln (c), Blindflanschen (d), Kreuzblenden (e), Befestigung (f), MiniTec-Gestell (g) mit Umlenkrollen, Schrittmotor mit Antriebsrad (h), Kunststoffschnur mit Spannfeder (i), Teflon-Störkörper (j).}
\end{sidewaysfigure}


%------------------------------------------------------------------------------
\section{Messmethodik}
%------------------------------------------------------------------------------
Fit der Resonanzkurve an die Güte\\
Resonanzfrequenz aus dem Minimum der Resonanzkurve
