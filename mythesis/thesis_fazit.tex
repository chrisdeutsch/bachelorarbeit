%==============================================================================
\chapter{Zusammenfassung und Ausblick}
\label{sec:fazit}
%==============================================================================
Im Rahmen dieser Arbeit wurde die Methode der resonanten Störkörpermessung zur Bestimmung der elektrischen Felder in Hohlraumresonatoren vorgestellt.
Dazu wurde ein Störkörpermessstand aufgebaut und um die Möglichkeit einer Störkörpermessung ohne Temperaturstabilisierung des Resonators erweitert.
Diese Messmethode wurde an zwei siebenzelligen Resonatoren vom Typ PETRA erfolgreich erprobt, indem eine Vermessung der elektrischen Felder der $\mathrm{TM}_{010}$-Resonatormode und einiger Moden höherer Ordnung erfolgte.

Die Vermessung der Fundamentalmode beider Resonatoren ergab effektive Shuntimpedanzen von etwa \SI{25}{\mega\ohm} und überschreitet somit die Erwartungen von \SI{23}{\mega\ohm} in \cite{schedler}.
Demnach sollten beide Resonatoren vom Standpunkt der Beschleunigungsspannung uneingeschränkt bei der Erweiterung von ELSA zum Einsatz kommen können.

Außerdem wurde eine Vermessung von einigen Moden höherer Ordnung durchgeführt und deren Shuntimpedanzen bestimmt.
Alle effektiven Shuntimpedanzen der vemessenen Moden höherer Ordnung liegen mehrere Größenordnungen unter der effektiven Shuntimpedanz der Fundamentalmode und stellen keine große Gefahr für die Anregung von Multi-Bunch-Instabilitäten dar.
Da keine umfassende Vermessung aller Moden höherer Ordnung durchgeführt wurde, kann der Zusammenhang von weiteren Moden mit Multi-Bunch-Instabilitäten an ELSA eine Grundlage für zukünftige Untersuchungen darstellen.
