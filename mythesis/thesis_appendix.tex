%------------------------------------------------------------------------------
\chapter{>>>Theorie<<<}
\label{sec:appendix}
%------------------------------------------------------------------------------


%------------------------------------------------------------------------------
\section{Herleitung der Frequenzverschiebung bei Störkörpermessung}
\label{app:herleitung_frequenzverschiebung}
%------------------------------------------------------------------------------
Man betrachte das zeit- und ortsabhängige elektromagnetische Feld in einem Hohlraumresonators, charakterisiert durch elektrische Feldstärke~$\ve$ und magnetische Feldstärke~$\vh$.
Die Felder des ungestörten Resonators seien gegeben durch $(\ve_0, \vh_0)$ und die des gestörten Resonators durch $(\ve_1, \vh_1)$.
Bei Verwendung der komplexen Darstellung der stehenden Welle im Resonator, kann das elektromagnetische Feld angegeben werden als:
\begin{subequations}
  \label{eq:skm_felder}
  \begin{align}
  &\ve_{0,1}(x,y,z,t) = \ve_{0,1}(x,y,z) \, e^{i \omega_{0,1} t}\\
  &\vh_{0,1}(x,y,z,t) = \vh_{0,1}(x,y,z) \, e^{i \omega_{0,1} t}
  \end{align}
\end{subequations}
wobei die Phasenbeziehung von elektrischem und magnetischem Feld in den komplexen Amplituden $\ve_{0,1}(x,y,z)$ und $\vh_{0,1}(x,y,z)$. enthalten ist.
Unabhängig von der Störung des Feldes gelten die \textsc{Maxwell}-Gleichungen \cite{jackson}:
\begin{subequations}
  \label{eq:skm_maxwell}
  \begin{align}
    \vnabla \times \ve &= - \frac{\partial \vb}{\partial t}\\
    \vnabla \times \vh &= \vec{j}_\mathrm{frei} + \frac{\partial \vd}{\partial t}
  \end{align}
\end{subequations}
Setzt man die gestörten und ungestörten Felder \eqref{eq:skm_felder} in die \textsc{Maxwell}-Gleichung \eqref{eq:skm_maxwell} ein, so erhält man unter der Annahme einer verschwindenden freien Stromdichte $\vec{j}_\mathrm{frei}$ im Hohlraum und nach Elimination der Zeitabhängigkeit:
\begin{subequations}
  \label{eq:skm_zeitunabhaengig}
  \begin{align}
    &\vnabla \times \ve_{0,1} = - i \omega_{0,1} \vb_{0,1} \\
    &\vnabla \times \vh_{0,1} = i \omega_{0,1} \vd_{0,1}
  \end{align}
\end{subequations}
Man verwendet diese zeitunabhängigen Gleichungen und die Produktregel der Divergenz für das Kreuzprodukt um die folgenden Identitäten zu finden:
\begin{subequations}
  \label{eq:skm_vektoridentitaeten}
  \begin{align}
  \vnabla \cdot \left( \ve_0^* \times \vh_1\right) &= \vh_1 \cdot \left( \vnabla \times \ve_0^* \right) - \ve_0^* \cdot \left( \vnabla \times \vh_1 \right) \nonumber \\
  &= i \omega_0 \vb_0^* \vh_1 - i \omega_1 \ve_0^* \vd_1 \label{eq:e0h1} \\[0.5em]
  %
  \vnabla \cdot \left( \ve_1 \times \vh_0^* \right) &= \vh_0^* \cdot \left( \vnabla \times \ve_1 \right) - \ve_1 \cdot \left( \vnabla \times \vh_0^* \right) \nonumber \\
  &= i \omega_0 \ve_1 \vd_0^* - i \omega_1 \vb_1 \vh_0^* \label{eq:e1h0}
  \end{align}
\end{subequations}
Anschließend bildet man die Summe der Gleichungen \eqref{eq:skm_vektoridentitaeten} und führt eine Integration über das Resonatorvolumen $V$ durch.
Unter Verwendung des \textsc{Gauß}schen Integralsatzes erhält man:
\begin{align}
  \int_{V} \mathrm{d}V \left[ \vnabla \cdot \left( \ve_0^* \times \vh_1 + \ve_1 \times \vh_0^* \right) \right] = \oint_{\partial V} \mathrm{d}S \left[ \vec{n} \cdot \left( \ve_0^* \times \vh_1 + \ve_1 \times \vh_0^* \right)\right] \label{eq:volint}
\end{align}
wobei $\vec{n}$ den Normaleneinheitsvektor auf dem Rand $\partial V$ des Resonatorhohlraums darstellt.
Die Randbedingungen für das elektrische und magnetische Feld am idealen Leiter \eqref{eq:randbedingung_leiter} sorgen dafür, dass das Skalarprodukt im Integranden der rechten Seite auf dem Rand des Volumens $\partial V$ identisch verschwindet\footnote{Das Kreuzprodukt von elektrischer und magnetischer Feldstärke steht stehts senkrecht zum Normalenvektor der ideal leitenden Grenzfläche: $\ve \times \vh \perp \vec{n}$}.
Setzt man die gefundenen Identitäten \eqref{eq:skm_vektoridentitaeten} in das Integral über das Resonatorvolumen ein, so erhält den Zusammenhang zwischen ungestörter~$\omega_0$ und gestörter Resonanzfrequenz~$\omega_1$:
\begin{align}
  \omega_0 \int_{V} \mathrm{d}V \left( \vb_0^* \cdot \vh_1 + \ve_1 \cdot \vd_0^* \right) = \omega_1 \int_{V} \mathrm{d}V \left( \vb_1 \cdot \vh_0^* + \ve_0^* \cdot \vd_1 \right)
\end{align}
Dieser Zusammenhang ermöglicht es einen Ausdruck ermöglicht die Angabe der relativen Frequenzverschiebung:
\begin{align}
  \frac{\Delta \omega}{\omega_0}= \frac{\omega_1 - \omega_0}{\omega_0} = \frac{\int_{V} \mathrm{d}V \left[ \left( \ve_1 \cdot \vd_0^* - \ve_0^* \cdot \vd_1 \right) + \left( \vb_0^* \cdot \vh_1 - \vb_1 \cdot \vh_0^* \right)\right]}{\int_V \mathrm{d}V \left[\ve_0^* \cdot \vd_1 + \vb_1 \cdot \vh_0^* \right] }
  \label{eq:skm_rel_freqabweichung_schritt}
\end{align}
Unter Verwendung der Definitionen für die magnetische Feldstärke $\vh$ und der elektrischen Flussdichte $\vd$:
\begin{subequations}
  \begin{align}
    &\vd \coloneqq \epsilon_0 \ve + \vec{P}\\
    &\vh \coloneqq \frac{1}{\mu_0} \vb - \vec{M}
  \end{align}
\end{subequations}
mit den Vektorfeldern der Polarisation $\vec{P}$ und Magnetisierung $\vec{M}$ folgt aus \eqref{eq:skm_rel_freqabweichung_schritt}:
\begin{align}
  \frac{\Delta \omega}{\omega_0} = - \frac{\int_V \mathrm{d}V \left[ \ve_0^* \cdot \vec{P} + \vb_0^* \cdot \vec{M} \right]}{\int_V \mathrm{d}V \left[ \ve_0^* \cdot \vd_1 + \vb_1 \cdot \vh_0^* \right]} \label{eq:skm_rel_freqabweichung_schritt2}
\end{align}
Schließlich wird angenommen, dass das Volumen des Störkörpers klein ist gegen das gesamte Resonatorvolumen.
Dadurch können bei Integration über das Resonatorvolumen im Nenner von \eqref{eq:skm_rel_freqabweichung_schritt2}, die gestörten Felder im Integranden durch die Ungestörten genähert werden.
Beachtet man weiterhin, dass die Phasendifferenz von elektrischem und magnetischem Feld bei stehenden elektromagnetischen Wellen $\pi / 2$ beträgt, so erhält man nach Ausführen der Integration im Nenner die vierfache im Feld des Resonators gespeicherte Energie~$W_0$:
\begin{align}
  \frac{\Delta \omega}{\omega_0} = - \frac{\int_V \mathrm{d}V \left[ \ve_0^* \cdot \vec{P} + \vb_0^* \cdot \vec{M} \right]}{4 W_0}
\end{align}
Diese Gleichung verknüpft die Verschiebung der Resonanzfrequenz durch einen Störkörper mit den ungestörten elektrischen und magnetischen Feldern und ermöglicht die Messung der Felder bei geeigneter Wahl des Störkörpers.
\todo{Frequenzverschiebung ist negativ}
